\section{Structure of spin space group}

\subsection{Spin space group}

A spin space group \cite{doi:10.1063/1.1708514,doi:10.1098/rspa.1966.0211,LITVIN1974538,Opechowski1986} is used to classify spin symmetry operations of spin arrangements.

\begin{screen}
  \begin{definition}[Spin space group]
    \label{def:spin_space_group}
    Let $\mathcal{G}$ be a subgroup of $\mathrm{E}(3) \times \mathrm{O}(3)$.
    When the following $\mathcal{F}(\mathcal{G})$ and $\mathcal{D}(\mathcal{G})$ are space groups, $\mathcal{G}$ is a \term{spin space group} \cite{LITVIN1974538},
    \begin{align}
      \mathcal{F}(\mathcal{G})
        &\coloneqq \set{ g \in \mathrm{E}(3) }{ \exists \bm{W} \in \mathrm{O}(3) \,s.t.\, (g, \bm{W}) \in \mathcal{G} } \\
      \mathcal{D}(\mathcal{G})
        &\coloneqq \set{ g \in \mathrm{E}(3) }{ (g, \bm{E}) \in \mathcal{G} }.
    \end{align}
    For a spin space group $\mathcal{G}$, we call $\mathcal{F}(\mathcal{G})$ a \term{family space group} and $\mathcal{D}(\mathcal{G})$ a \term{maximal space subgroup}.
  \end{definition}
\end{screen}
Reference~\cite{LITVIN1974538} did not impose the condition that $\mathcal{D}(\mathcal{G})$ is crystallographic.
However, this condition seems to be required to guarantee a spin translation group $\mathcal{G}_{\mathrm{st}}(\mathcal{G})$ not to be subperiodic.

\begin{screen}
  \begin{definition}[Spin-space-group type]
    \label{def:spin_space_group_type}
    Two spin space groups $\mathcal{G}_{1}$ and $\mathcal{G}_{2}$ belong to the same \term{spin-space-group type} if they are transformed to the other by an orientation-preserving transformation:
    \begin{align}
      \mathcal{G}_{1} \sim \mathcal{G}_{2}
      \overset{\mathrm{def}}{\Longleftrightarrow}
      \exists (\bm{P}, \bm{p}) \,s.t.\, (\bm{P}, \bm{p})^{-1} \mathcal{G}_{1} (\bm{P}, \bm{p}) = \mathcal{G}_{2},
    \end{align}
    where we define the action of transformations to a spin symmetry $(g, \bm{W})$ as
    \begin{align}
      (\bm{P}, \bm{p})^{-1} (g, \bm{W}) (\bm{P}, \bm{p})
      \coloneqq
      ((\bm{P}, \bm{p})^{-1} g (\bm{P}, \bm{p}), \bm{P}^{-1}\bm{W}\bm{P}).
    \end{align}
  \end{definition}
\end{screen}

\subsection{Spin-only group}

\begin{screen}
  \begin{definition}[Spin-only group]
    Let $\mathcal{G}$ be a spin space group.
    A \term{spin-only group} of $\mathcal{G}$ is
    \begin{align}
      \mathcal{P}_{\mathrm{so}}(\mathcal{G})
      \coloneqq
      \set{ \bm{W} \in \mathrm{O}(3) }{ ((\bm{E}, \bm{0}), \bm{W}) \in \mathcal{G} }.
    \end{align}
  \end{definition}
\end{screen}

A spin-only group $\mathcal{P}_{\mathrm{so}}(\mathcal{G})$ is a normal subgroup of $\mathcal{G}$, $1 \times \mathcal{P}_{\mathrm{so}}(\mathcal{G}) \trianglelefteq \mathcal{G}$.
A \term{nontrivial spin space group} of $\mathcal{G}$ is defined as $\mathcal{G} / (1 \times \mathcal{P}_{\mathrm{so}}(\mathcal{G}))$.

When a spin space group $\mathcal{G}$ is a stabilizer of a spin arrangement, spin-only groups are classified as Table~\ref{tab:spin_only_group} \cite{LITVIN1974538,PhysRevX.12.021016}.

\begin{table}[tb]
  \centering
  \caption{Classification of spin-only groups}
  \label{tab:spin_only_group}
  \begin{tabular}{cc}
    \hline \hline
    Spin-only group & Spin arrangement \\
    \hline
    $\infty \infty m \cong \mathrm{O}(3)$ & Nonmagnetic \\
    $\infty m \cong \mathrm{SO}(2) \rtimes \mathbb{Z}_{2} $ & Collinear \\
    $m$ & Coplanar \\
    $1$ & Noncoplaner \\
    \hline \hline
  \end{tabular}
\end{table}

\subsection{Spin translation group}

\begin{screen}
  \begin{definition}[Spin translation group]
    Let $\mathcal{G}$ be a spin space group.
    A \term{spin translation group} of $\mathcal{G}$ is
    \begin{align}
      \mathcal{G}_{\mathrm{st}}(\mathcal{G})
      \coloneqq
      \set{ ((\bm{E}, \bm{v}), \bm{W}) }{ ((\bm{E}, \bm{v}), \bm{W}) \in \mathcal{G}}.
    \end{align}
  \end{definition}
\end{screen}

Reference~\cite{Litvin:a09793} classified the spin translation groups.
A \term{nontrivial spin translation group} of $\mathcal{G}$ is defined as $\mathcal{G}_{\mathrm{st}}(\mathcal{G}) / (1 \times \mathcal{P}_{\mathrm{so}}(\mathcal{G}))$.

We write a translation subgroup and a point group of space group $\mathcal{S}$ as $\mathcal{T}(\mathcal{S})$ and $\mathcal{P}(\mathcal{S})$, respectively,
\begin{align}
  \mathcal{T}(\mathcal{S}) &\coloneqq \set{ (\bm{E}, \bm{t}) }{ (\bm{E}, \bm{t}) \in \mathcal{S} } \\
  \mathcal{P}(\mathcal{S}) &\coloneqq \set{ \bm{W} }{ \exists \bm{v} \, s.t.\, (\bm{W}, \bm{v}) \in \mathcal{S} }.
\end{align}

The spin-only group $\mathcal{P}_{\mathrm{so}}(\mathcal{G})$ is a normal subgroup of $\mathcal{G}_{\mathrm{st}}(\mathcal{G})$.
A translation subgroup of $\mathcal{D}(\mathcal{G})$ is also a normal subgroup of $\mathcal{G}_{\mathrm{st}}(\mathcal{G})$.
Because $(1 \times \mathcal{P}_{\mathrm{so}}(\mathcal{G})) \cap (\mathcal{T}(\mathcal{D}(\mathcal{G})) \times 1) = \{ 1 \}$, we can consider a factor group $\mathcal{G}_{\mathrm{st}}(\mathcal{G}) / ( \mathcal{T}(\mathcal{D}(\mathcal{G})) \times \mathcal{P}_{\mathrm{so}}(\mathcal{G}))$.
We write one of the coset decompositions as
\begin{align}
  \label{eq:spin_translation_group}
  \mathcal{G}_{\mathrm{st}}(\mathcal{G})
    &= \bigsqcup_{ \bm{v} }
        ((\bm{E}, \bm{v}), \bm{W}_{\bm{v}})
        \left(
          \mathcal{T}(\mathcal{D}(\mathcal{G}))
          \times
          \mathcal{P}_{\mathrm{so}}(\mathcal{G})
        \right),
\end{align}
where we choose $\bm{W}_{\bm{0}} = \bm{E}$.
Because both $\mathcal{T}(\mathcal{D}(\mathcal{G}))$ and $\mathcal{T}(\mathcal{F}(\mathcal{G}))$ are three-dimensional translation subgroups, a factor group $\mathcal{T}(\mathcal{F}(\mathcal{G})) / \mathcal{T}(\mathcal{D}(\mathcal{G}))$ is finite.
Thus, a set of the coset representatives $\{ ((\bm{E}, \bm{v}), \bm{W}_{\bm{v}}) \}_{\bm{v}}$ is finite.
Be careful that $\bm{W}_{\bm{v}}$ may not be a crystallographic rotation in general!

\subsection{Nontrivial spin point group}

Because $\mathcal{G}_{\mathrm{st}}(\mathcal{G})$ is a normal subgroup of $\mathcal{G}$, we can consider a factor group $\mathcal{G} / \mathcal{G}_{\mathrm{st}}(\mathcal{G})$.
\begin{screen}
  \begin{definition}[Nontrivial spin point group]
    Let $\mathcal{G}$ be a spin space group.
    A \term{nontrivial spin point group} of $\mathcal{G}$ is a factor group
    \begin{align}
      \mathcal{U}_{\mathrm{np}}(\mathcal{G}) \coloneqq \mathcal{G} / \mathcal{G}_{\mathrm{st}}(\mathcal{G}).
    \end{align}
  \end{definition}
\end{screen}

We denote its corresponding coset decomposition,
\begin{align}
  \label{eq:spin_space_group_coset}
  \mathcal{G}
    = \bigsqcup_{ \bm{R} \in \mathcal{P}(\mathcal{F}(\mathcal{G})) } ((\bm{R}, \bm{v}_{\bm{R}}), \bm{W}_{\bm{R}}) \mathcal{G}_{\mathrm{st}}(\mathcal{G}),
\end{align}
where we choose $\bm{W}_{\bm{E}} = \bm{E}$ and $\bm{v}_{\bm{E}} = \bm{0}$.

The nontrivial spin point group $\mathcal{U}_{\mathrm{np}}(\mathcal{G})$ is also finite because point group $\mathcal{P}(\mathcal{F}(\mathcal{G}))$ is finite,
Because the map $\bm{W}_{\bullet}: \mathcal{P}(\mathcal{F}(\mathcal{G})) \to \mathrm{O}(3)$ is linear, there exists a normal subgroup $N \trianglelefteq \mathcal{P}(\mathcal{F}(\mathcal{G}))$ such that $\{ \bm{W}_{\bm{R}} \}_{ \bm{R} \in \mathcal{P}(\mathcal{F}(\mathcal{G})) } \cong \mathcal{P}(\mathcal{F}(\mathcal{G})) / N$.
Thus, spin point groups are classified by 32 crystallographic point groups and their normal subgroups \cite{Litvin:a14103}.

\subsection{Spin point group}

\todo{
    In this subsection, we assume a spin space group $\mathcal{G}$ is a stabilizer of a spin arrangement.
    Thus, $\mathcal{P}_{\mathrm{so}}(\mathcal{G})$ is isomorphic to $1$, $m$, $\infty m$, or $\infty \infty m$.
}

We define a \term{spin point group} of a spin space group $\mathcal{G}$ as
\begin{align}
    \mathcal{U}(\mathcal{G}) \coloneqq \set{ (\bm{R}, \bm{W}) }{ \exists \bm{v} \,s.t.\, ((\bm{R}, \bm{v}), \bm{W}) \in \mathcal{G} }.
\end{align}

A \term{family spin point group} of $\mathcal{G}$ is defined as
\begin{align}
  \mathcal{B}(\mathcal{G})
    &\coloneqq \set{ \bm{W} }{ (\bm{R}, \bm{W}) \in \mathcal{U}(\mathcal{G}) }.
\end{align}

\subsubsection{Family spin point group of spin translation group}

We write a family spin point group of a spin translation group in Eq.~\eqref{eq:spin_translation_group} as
\begin{align}
    \mathcal{B}_{\mathrm{st}}(\mathcal{G})
        &\coloneqq \mathcal{B}(\mathcal{G}_{\mathrm{st}}(\mathcal{G}))
        = \bigsqcup_{ \bm{v} } \bm{W}_{\bm{v}} \mathcal{P}_{\mathrm{so}}(\mathcal{G}).
\end{align}
Since $\mathcal{B}_{\mathrm{st}}: \mathcal{G} \to \mathcal{B}_{\mathrm{st}}(\mathcal{G})$ is a surjective group homomorphism, $\mathcal{P}_{\mathrm{so}}(\mathcal{G})$ is also a normal subgroup of $\mathcal{B}_{\mathrm{st}}(\mathcal{G})$.

Let $\mathcal{B}$ be a subgroup of a normalizer of $\mathcal{P}_{\mathrm{so}}(\mathcal{G})$ in $\mathrm{O}(3)$.
Then, we prove $\mathcal{B}$ is an internal direct product with $\mathcal{P}_{\mathrm{so}}(\mathcal{G})$ as follows.
The normalizers of spin-only groups are classified in Table~\ref{tab:spin_only_normalizers}.
For the noncoplanar case, $\mathcal{B} = 1 \times \mathcal{B} = \mathcal{P}_{\mathrm{so}} \times \mathcal{B}$ is trivially an internal direct product.
For the collinear case, $\mathcal{B}$ is $\mathcal{P}_{\mathrm{so}}$ or $\mathcal{P}_{\mathrm{so}} \times C_{1h}$, which are both internal direct products.
For the coplanar case, $\mathcal{B}$ should be one of the following (see Tables 3.2.1.3 and 3.2.1.6 of Ref.~\cite{hahn2016point}),
\begin{itemize}
  \item $C_{1h}$
  \item $C_{nh} = C_{1h} \times C_{n}$ ($n \geq 2$, $n$ is even)
  \item $D_{nh} = C_{1h} \times D_{n}$ ($n \geq 2$)
  % C_1h is not normal in T_h, O_h, and I_h.
  \item $C_{\infty h} = C_{1h} \times C_{\infty}$
  \item $D_{\infty h} = C_{1h} \times D_{\infty}$.
\end{itemize}
Thus, $\mathcal{B}$ is an internal direct product.
For the nonmagnetic case, $\mathcal{B} = \mathcal{P}_{\mathrm{so}}$ is trivially an internal direct product.

% Normalizers of spin-only groups
\begin{table}[tb]
  \centering
  \caption{
    Normalizers of spin-only groups based on Table~3.5.4.2 of Ref.~\cite{koch2016normalizers}.
  }
  \label{tab:spin_only_normalizers}
  \begin{tabular}{ccc}
    \hline \hline
    Spin arrangement & Spin-only group $\mathcal{P}_{\mathrm{so}}$ & Normalizer \\
    \hline
    Nonmagnetic & $\infty \infty m \,(\mathrm{O}(3))$ & $\infty \infty m \,(\mathrm{O}(3))$ \\
    Collinear & $\infty m \,(C_{\infty v})$ & $\infty / mm \, (D_{\infty h})$ \\
    Coplanar & $m \,(C_{1h})$ & $\infty / mm \, (D_{\infty h})$ \\
    Noncoplanar & $1 \, (C_{1})$ & $\infty \infty m \,(\mathrm{O}(3))$ \\
    \hline \hline
  \end{tabular}
\end{table}

Therefore, there exists a \term{nontrivial family spin point group of spin translation group} $\bar{\mathcal{B}}_{\mathrm{st}}(\mathcal{G}) \trianglelefteq \mathrm{O}(3)$ such that $\mathcal{B}_{\mathrm{st}}(\mathcal{G}) = \mathcal{P}_{\mathrm{so}}(\mathcal{G}) \times \bar{\mathcal{B}}_{\mathrm{st}}(\mathcal{G})$.
Furthermore, $\bar{\mathcal{B}}_{\mathrm{st}}(\mathcal{G})$ is abelian because the map $\bm{v} \mapsto \bm{W}_{\bm{v}}$ in Eq.~\eqref{eq:spin_translation_group} is a surjective group homomorphism and a translation group is abelian.
Abelian general point groups are listed in Table~\ref{tab:abelian_point_groups}.

\subsubsection{\label{sec:family_spin_point_group}Family spin point group of spin space group}

% Family spin point group: B
Since $\mathcal{P}_{\mathrm{so}}(\mathcal{G})$ is a normal subgroup of $\mathcal{B}(\mathcal{G})$, $\mathcal{B}(\mathcal{G})$ is an internal direct product with $\mathcal{P}_{\mathrm{so}}(\mathcal{G})$ with a similar discussion in the previous section,
\begin{align}
  \mathcal{B}(\mathcal{G})
    &=: \mathcal{P}_{\mathrm{so}}(\mathcal{G}) \times \bar{\mathcal{B}}(\mathcal{G}).
\end{align}
Because $\bar{\mathcal{B}}_{\mathrm{st}}(\mathcal{G})$ is an \textbf{abelian normal subgroup} of $\bar{\mathcal{B}}(\mathcal{G})$, $\bar{\mathcal{B}}(\mathcal{G})$ is an internal semidirect product with $\bar{\mathcal{B}}_{\mathrm{st}}(\mathcal{G})$,
\begin{align}
    \bar{\mathcal{B}}(\mathcal{G})
        &=: \bar{\mathcal{B}}_{\mathrm{st}}(\mathcal{G}) \rtimes \bar{\mathcal{B}}_{\mathrm{np}}(\mathcal{G}) \\
    \therefore\quad \mathcal{B}(\mathcal{G})
        &= \mathcal{P}_{\mathrm{so}}(\mathcal{G})
        \times
        \left(
          \bar{\mathcal{B}}_{\mathrm{st}}(\mathcal{G}) \rtimes \bar{\mathcal{B}}_{\mathrm{np}}(\mathcal{G})
        \right)
\end{align}

\subsubsection{Spin point group of spin space group}

A \term{family point group} of $\mathcal{G}$ is defined as
\begin{align}
  \mathcal{R}(\mathcal{G})
    &\coloneqq \set{ \bm{R} }{ \exists \bm{v}, \bm{W} \,s.t.\, ((\bm{R}, \bm{v}), \bm{W}) \in \mathcal{G} }
    = \mathcal{P}(\mathcal{F}(\mathcal{G})).
\end{align}
A \term{maximal point subgroup} of $\mathcal{G}$ is defined as
\begin{align}
  \mathcal{K}(\mathcal{G})
    &\coloneqq \set{ \bm{R} }{ \exists \bm{v} \,s.t.\, ((\bm{R}, \bm{v}), \bm{E}) \in \mathcal{G} }.
\end{align}

We can construct $\mathcal{U}(\mathcal{G})$ as an internal semidirect products with the coset representatives in Eq.~\eqref{eq:spin_space_group_coset}:
\begin{align}
  \mathcal{U}(\mathcal{G})
    &= \bigsqcup_{ \bm{R} \in \mathcal{R}(\mathcal{G}) } (\bm{R}, \bm{W}_{\bm{R}}) \left( \mathcal{P}_{\mathrm{so}}(\mathcal{G}) \times \bar{\mathcal{B}}_{\mathrm{st}}(\mathcal{G}) \right) \nonumber \\
    &= \mathcal{P}_{\mathrm{so}}(\mathcal{G})
        \times \left(
          \bar{\mathcal{B}}_{\mathrm{st}}(\mathcal{G})
          \rtimes
          \mathcal{U}_{\mathrm{np}}(\mathcal{G})
         \right)
       \quad (\because \mbox{ $\mathcal{P}_{\mathrm{so}}(\mathcal{G}), \bar{\mathcal{B}}_{\mathrm{st}}(\mathcal{G}) \trianglelefteq \mathcal{U}(\mathcal{G})$, $\bar{\mathcal{B}}_{\mathrm{st}}(\mathcal{G})$ is abelian } ) \nonumber \\
    &= \mathcal{P}_{\mathrm{so}}(\mathcal{G})
        \times \left(
          \bar{\mathcal{B}}_{\mathrm{st}}(\mathcal{G})
          \rtimes
          \left(
            \mathcal{K}(\mathcal{G})
            \rtimes
            \bar{\mathcal{B}}_{\mathrm{np}}(\mathcal{G})
          \right)
         \right)
      \quad (\because \mathcal{K}(\mathcal{G}) \cap \bar{\mathcal{B}}_{\mathrm{np}}(\mathcal{G}) = \{ 1 \})
\end{align}
where we can choose the nontrivial spin point group $\mathcal{U}_{\mathrm{np}}(\mathcal{G})$ as a subgroup of $\mathcal{U}(\mathcal{G})$,
\begin{align}
  \mathcal{U}_{\mathrm{np}}(\mathcal{G})
    &= \left\{ (\bm{R}, \bm{W}_{\bm{R}}) \right\}_{ \bm{R} \in \mathcal{R}(\mathcal{G}) } \nonumber \\
    &= \mathcal{K}(\mathcal{G}) \rtimes \bar{\mathcal{B}}_{\mathrm{np}}(\mathcal{G}) \nonumber \\
    &= \bigsqcup_{ \bm{S} \in \mathcal{R}(\mathcal{G}) / \mathcal{K}(\mathcal{G}) } (\bm{S}, \bm{W}_{\bm{S}}) \mathcal{K}(\mathcal{G}).
\end{align}


\subsubsection{Enumeration of spin point group types}

\begin{enumerate}
  \item Enumerate family spin point groups $\mathcal{B} = \mathcal{P}_{\mathrm{so}} \times ( \bar{\mathcal{B}}_{\mathrm{st}} \rtimes \bar{\mathcal{B}}_{\mathrm{np}} )$ up to transformation
    \begin{enumerate}
      \item Choose spin-only group $\mathcal{P}_{\mathrm{so}}$: $1$, $m$, $\infty m$, $\infty \infty m$
      \item Choose a group $\mathcal{B}$ such that $\mathcal{P}_{\mathrm{so}} \trianglelefteq \mathcal{B} \leq \mathcal{N}(\mathcal{P}_{\mathrm{so}})$
      \item Choose a group $\bar{\mathcal{B}}$ such that $\bar{\mathcal{B}} \trianglelefteq \mathcal{B}$, $\mathcal{P}_{\mathrm{so}} \cap \bar{\mathcal{B}} = \{ 1 \}$, and $\mathcal{P}_{\mathrm{so}} \bar{\mathcal{B}} = \mathcal{B}$: $\mathcal{B} = \mathcal{P}_{\mathrm{so}} \times \bar{\mathcal{B}}$
      \item Choose an \textbf{abelian} group $\bar{\mathcal{B}}_{\mathrm{st}}$ such that $\bar{\mathcal{B}}_{\mathrm{st}} \trianglelefteq \bar{\mathcal{B}}$
      \item Choose a crystallographic point group $\bar{\mathcal{B}}_{\mathrm{np}}$ such that $\bar{\mathcal{B}}_{\mathrm{np}} \leq \bar{\mathcal{B}}$ and $\bar{\mathcal{B}}_{\mathrm{st}} \cap \bar{\mathcal{B}}_{\mathrm{np}} = \{ 1 \}$: $\bar{\mathcal{B}} = \bar{\mathcal{B}}_{\mathrm{st}} \rtimes \bar{\mathcal{B}}_{\mathrm{np}}$
    \end{enumerate}
  \item Enumerate crystallographic family point group $\mathcal{R}$ up to geometric crystal class
  \item Enumerate maximal point group $\mathcal{K} \trianglelefteq \mathcal{R}$ up to conjugacy
  \item Enumerate semidirect products $\mathcal{K} \rtimes \bar{\mathcal{B}}_{\mathrm{np}}$ from Litvin's table
  \item Enumerate semidirect products $\bar{\mathcal{B}}_{\mathrm{st}} \rtimes (\mathcal{K} \rtimes \bar{\mathcal{B}}_{\mathrm{np}})$
  \item (We can construct spin point group $\mathcal{U} = \mathcal{P}_{\mathrm{so}} \times \left( \bar{\mathcal{B}}_{\mathrm{st}} \rtimes (\mathcal{K} \rtimes \bar{\mathcal{B}}_{\mathrm{np}}) \right)$)
\end{enumerate}


\begin{itemize}
  \item Noncoplanar: Table~\ref{tab:spin_point_group_noncoplanar}
  \item Coplanar: Table~\ref{tab:spin_point_group_coplanar}
  \item Collinear: Table~\ref{tab:spin_point_group_collinear_nonmagnetic}
\end{itemize}

% Spin point groups (with spin-only groups)
\begin{table}[tb]
  \centering
  \caption{
    \todo{}
    Enumeration of spin point groups for noncoplanar.
  }
  \label{tab:spin_point_group_noncoplanar}
  \begin{tabular}{ccc}
    \hline \hline
    $\bar{\mathcal{B}}_{\mathrm{st}}(\mathcal{G}) \rtimes \bar{\mathcal{B}}_{\mathrm{np}}(\mathcal{G})$  % Compliment of P_{so}, which is a subgroup of N(P_{so})
        & $\bar{\mathcal{B}}_{\mathrm{st}}(\mathcal{G})$  % Abelian
        & $\bar{\mathcal{B}}_{\mathrm{np}}(\mathcal{G})$  % Crystallographic
        \\
    \hline
   * & $1$ & * \\
   & $2$ & \\
   & $m$ & \\
   & $\overline{1}$ & \\
   & $C_{n} = n \, (n \geq 3)$ & \\
   & $S_{n} = \overline{n} \, (n \geq 4, \mbox{$n$ is even})$ & \\
   & $C_{nh} = n/m \, (n \geq 2, \mbox{$n$ is even})$ & \\
   & $C_{nh} = \overline{n} \, (n \geq 3, \mbox{$n$ is odd})$ & \\
   & $C_{2v} = mm2$ & \\
   & $D_{2} = 222$ & \\
   & $D_{2h} = mmm$ & \\
   & $C_{\infty} = \infty$ & \\
   & $C_{\infty h} = \infty / m$ & \\
    \hline \hline
  \end{tabular}
\end{table}

\begin{table}[tb]
  \centering
  \caption{
    Enumeration of family spin point groups for commensurate coplanar, $\mathcal{P}_{\mathrm{so}} = m, \mathcal{N}(\mathcal{P}_{\mathrm{so}}) = D_{\infty h}$.
  }
  \label{tab:spin_point_group_coplanar}
  \begin{tabular}{cc}
    \hline \hline
    $\mathcal{B} = m \times \bar{\mathcal{B}}$
        & $\bar{\mathcal{B}}$
        \\
    \hline
    $m$                                 & $1$            \\
    $C_{2h}$                            & $2$            \\
                                        & $\overline{1}$ \\
    $C_{nh} \, (n \geq 4, \mbox{even})$ & $C_{n}$        \\
                                        & $S_{n}$        \\
    $D_{nh} \, (n \geq 2)$              & $C_{nv}$       \\
                                        & $D_{n}$        \\
    % $C_{\infty h}$                      & $C_{\infty}$ & & \\
    % $D_{\infty h}$                      & $D_{\infty}$ & & \\
    \hline \hline
  \end{tabular}
\end{table}

\begin{table}[tb]
  \centering
  \caption{
    Enumeration of family spin point groups for commensurate coplanar with $\bar{\mathcal{B}}_{\mathrm{st}} = 1$.
  }
  \label{tab:spin_point_group_coplanar_st1}
  \begin{tabular}{cc}
    \hline \hline
    $\mathcal{B} = m \times \bar{\mathcal{B}}$
        & $\bar{\mathcal{B}} = \bar{\mathcal{B}}_{\mathrm{np}}$  % Crystallographic
        % & $\bar{\mathcal{B}}_{\mathrm{st}}(\mathcal{G}) \rtimes \bar{\mathcal{B}}_{\mathrm{np}}(\mathcal{G})$  % Compliment of P_{so}, which is a subgroup of N(P_{so})
        \\
    \hline
    $m$                          & $1$            \\
    $C_{2h}$                     & $2$            \\
                                 & $\overline{1}$ \\
    $C_{4h}$                     & $C_{4}$        \\
                                 & $S_{4}$        \\
    $D_{nh} \, (n = 2, 3, 4, 6)$ & $C_{nv}$       \\
                                 & $D_{n}$        \\
    % & & $1$ & \\
    % & & $\overline{1}$ & \\
    % & & $C_{n} = n \, (n \geq 3)$ & \\
    % & & $S_{n} = \overline{n} \, (n \geq 4, \mbox{$n$ is even})$ & \\
    % & & $C_{nh} = \overline{n} \, (n \geq 3, \mbox{$n$ is odd})$ & \\
    % & & $C_{2v} = mm2$ & \\
    % & & $D_{2} = 222$ & \\
    % & & $C_{\infty} = \infty$ & \\
    \hline \hline
  \end{tabular}
\end{table}

\begin{table}[tb]
  \centering
  \caption{
    \todo{}
    Enumeration of family spin point groups for commensurate coplanar with $\bar{\mathcal{B}}_{\mathrm{st}} = 2$.
  }
  \label{tab:spin_point_group_coplanar_st2}
  \begin{tabular}{ccc}
    \hline \hline
    $\mathcal{B} = m \times \bar{\mathcal{B}}$
        & $\bar{\mathcal{B}} = 2 \times \bar{\mathcal{B}}_{\mathrm{np}}$
        & $\bar{\mathcal{B}}_{\mathrm{np}}$
        \\
    \hline
    $C_{2h}$                                   & $2$      & $1$       \\
    $C_{nh} \, (n = 4, 6, 8, 12)$              & $C_{n}$  & $C_{\frac{n}{2}}$ \\
                                               % & $S_{n}$  &       \\  Not direct product
    $D_{nh} \, (n \geq 2, \mbox{$n$ is even})$ & $C_{nv}$ &       \\
                                               & $D_{n}$  &       \\
    % $C_{\infty h}$              & $C_{\infty}$ & & \\
    % $D_{\infty h}$              & $D_{\infty}$ & & \\
    \hline \hline
  \end{tabular}
\end{table}

\begin{table}[tb]
  \centering
  \caption{
    Enumeration of family spin point groups for collinear and nonmagnetic cases.
  }
  \label{tab:spin_point_group_collinear_nonmagnetic}
  \begin{tabular}{ccccc}
    \hline \hline
    $\mathcal{P}_{\mathrm{so}}$  % Nonmagnetic, collinear, coplanar, noncoplanar
        & $\mathcal{B} = \mathcal{P}_{\mathrm{so}} \times \bar{\mathcal{B}}$
        & $\bar{\mathcal{B}}$  % Compliment of P_{so}, which is a subgroup of N(P_{so})
        & $\bar{\mathcal{B}}_{\mathrm{st}}(\mathcal{G})$  % Abelian
        & $\bar{\mathcal{B}}_{\mathrm{np}}(\mathcal{G})$  % Crystallographic
        \\
    \hline
    $\infty m$ & $\infty m$              & $1$            & $1$            & $1$ \\
               & $\mathbf{\infty / m m}$ & $\textcolor{red}{m}$ (Missing in LLHWL2022?)& $m$            & $1$ \\
               &                         &                & $1$            & $m$ \\
               &                         & $\overline{1}$ & $\overline{1}$ & $1$ \\
               &                         &                & $1$            & $\overline{1}$ \\
    \hline
    $\infty \infty m$ & $\mathbf{\infty \infty m}$ & $1$ & $1$ & $1$ \\
    \hline \hline
  \end{tabular}
\end{table}
