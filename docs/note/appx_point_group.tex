\section{\label{sec:appx_point_group}General point group}

Table~\ref{tab:point_groups} summarizes general point groups in three dimensions.

% General point groups
\begin{table}[tb]
  \centering
  \caption{
    General point groups in three dimensions, based on Tables 3.2.1.6 of Ref.~\cite{hahn2016point}
  }
  \label{tab:point_groups}
  \begin{tabular}{cccc}
    \hline \hline
    & Schoenflies & Hellmann-Mauguin & Order \\
    \hline
    % Involutional
    & $C_{1}$ & $1$                     & $1$ \\
    & $C_{2} \equiv D_{1}$ & $2$                     & $2$ \\
    & $C_{1h} \equiv C_{1v} \equiv S_{1}$ & $m \equiv \overline{2}$ & $2$ \\
    & $C_{i} \equiv S_{2}$ & $\overline{1}$          & $2$ \\
    \hline
    % Cyclic
    $n \geq 3$                     & $C_{n}$               & $n$            & $n$ \\
    $n \geq 4$, \mbox{$n$ is even} & $S_{n}$               & $\overline{n}$ & $n$ \\
    $n \geq 2$, \mbox{$n$ is even} & $C_{nh}$              & $n / m$        & $2n$ \\
    $n \geq 3$, \mbox{$n$ is odd}  & $C_{nh} \equiv S_{n}$ & $\overline{n}$ & $2n$ \\
    $n \geq 2$                     & $C_{nv}$              & $nmm (\mbox{$n$ is even})$, $nm (\mbox{$n$ is odd})$ & $2n$ \\
    \hline
    % Dihedral
    $n \geq 2$ & $D_{n}$  & $n22 (\mbox{$n$ is even})$, $n2 (\mbox{$n$ is odd})$                        & $2n$ \\
    $n \geq 2$ & $D_{nd}$ & $\overline{n}2m (\mbox{$n$ is even})$, $\overline{n}m (\mbox{$n$ is odd})$  & $4n$ \\
    $n \geq 2$ & $D_{nh}$ & $n/mmm (\mbox{$n$ is even})$, $\overline{2n}2m (\mbox{$n$ is odd})$         & $4n$ \\
    \hline
    % Polyhedral
    & $T$     & $23$             & $12$ \\
    & $T_{h}$ & $m\overline{3}$  & $24$ \\
    & $T_{d}$ & $\overline{4}3m$ & $24$ \\
    & $O$     & $432$            & $24$ \\
    & $O_{h}$ & $m\overline{3}m$ & $48$ \\
    & $I$     & $235$            & $60$ \\
    & $I_{h}$ & $m\overline{35}$ & $120$ \\
    \hline
    % Continuous
    & $C_{\infty}$                       & $\infty$          & $\infty$ \\
    & $C_{\infty h} \equiv S_{\infty}$   & $\infty / m$      & $\infty$ \\
    & $C_{\infty v}$                     & $\infty m$        & $\infty$ \\
    & $D_{\infty}$                       & $\infty 2$        & $\infty$ \\
    & $D_{\infty h} \equiv D_{\infty d}$ & $\infty / mm$     & $\infty$\\
    & $K$                                & $\infty \infty$   & $\infty$\\
    & $K_{h}$                            & $\infty \infty m$ & $\infty$\\
    \hline \hline
  \end{tabular}
\end{table}

% Abelian general point groups
\begin{table}[tb]
  \centering
  \caption{
    \todo{}
    Abelian general point groups in three dimensions.
  }
  \label{tab:abelian_point_groups}
  \begin{tabular}{cc}
    \hline \hline
    & Schoenflies \\
    \hline
    % Involutional
    & $C_{1}$                             \\
    & $C_{2} \equiv D_{1}$                \\
    & $C_{1h} \equiv C_{1v} \equiv S_{1}$ \\
    & $C_{i} \equiv S_{2}$                \\
    \hline
    % Cyclic
    $n \geq 3$                     & $C_{n}$               \\
    $n \geq 4$, \mbox{$n$ is even} & $S_{n}$               \\
    $n \geq 2$, \mbox{$n$ is even} & $C_{nh}$              \\
    $n \geq 3$, \mbox{$n$ is odd}  & $C_{nh} \equiv S_{n}$ \\
                                   & $C_{2v}$              \\
    \hline
    % Dihedral
    & $D_{2}$  \\
    & $D_{2h}$ \\
    \hline
    % Continuous
    & $C_{\infty}$                     \\
    & $C_{\infty h} \equiv S_{\infty}$ \\
    \hline \hline
  \end{tabular}
\end{table}

\subsection{Commutation of two rotations}

A rotation of angle $\theta \, (0 \leq \theta < 2 \pi)$ along a unit vector $\hat{\bm{n}}$ can be represented by a quaternion
\begin{align}
    \bm{q} = \cos \frac{\theta}{2} + (\hat{n}_{x} \bm{i} + \hat{n}_{y} \bm{j} + \hat{n}_{z} \bm{k}) \sin \frac{\theta}{2}.
\end{align}
Note that $\bm{q}$ and $-\bm{q}$ represent the same rotation.

Consider two quaternions for rotations,
\begin{align}
    \bm{p} &= p_{0} + p_{1} \bm{i} + p_{2} \bm{j} + p_{3} \bm{k} \quad (\norm{p} = 1) \\
    \bm{q}_{0} &= \cos \frac{\theta}{2} + \bm{i} \sin \frac{\theta}{2}.
\end{align}
The condition that the rotations are commutative is $\bm{p}\bm{q}_{0} = \pm \bm{q}_{0} \bm{p}$.

By comparing $\bm{p}\bm{q}_{0}$ and $\bm{q}_{0} \bm{p}$,
\begin{align}
    \bm{p}\bm{q}_{0} &= p_{0} \cos \frac{\theta}{2} - p_{1} \sin \frac{\theta}{2}
                        + \left( p_{0} \sin \frac{\theta}{2} + p_{1} \cos \frac{\theta}{2} \right) \bm{i} \nonumber \\
                        &+ \left( p_{3} \sin \frac{\theta}{2} + p_{2} \cos \frac{\theta}{2} \right) \bm{j}
                        + \left( -p_{2} \sin \frac{\theta}{2} + p_{3} \cos \frac{\theta}{2} \right) \bm{k} \\
    \bm{q}_{0}\bm{p} &= p_{0} \cos \frac{\theta}{2} - p_{1} \sin \frac{\theta}{2}
                        + \left( p_{0} \sin \frac{\theta}{2} + p_{1} \cos \frac{\theta}{2} \right) \bm{i} \nonumber \\
                        &+ \left( -p_{3} \sin \frac{\theta}{2} + p_{2} \cos \frac{\theta}{2} \right) \bm{j}
                        + \left( p_{2} \sin \frac{\theta}{2} + p_{3} \cos \frac{\theta}{2} \right) \bm{k}
\end{align},
we obtain the following three cases to be commutative:
\begin{itemize}
    \item $\theta = 0$ ($\bm{q}_{0}$ is identity)
    \item Axes for $\bm{p}$ and $\bm{q}_{0}$ are parallel (the two rotations are coaxial)
    \item $\theta = \pi$ and the axis for $\bm{p}$ is perpendicular to that for $\bm{q}_{0}$
\end{itemize}
