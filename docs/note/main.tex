\documentclass[a4paper, 11pt]{article}

% add warnings
\RequirePackage[l2tabu, orthodox]{nag}
\usepackage[all, warning]{onlyamsmath}

%codification of the document
\usepackage[utf8]{inputenc}
% font
\usepackage{newtxtext}  % Times and Helvetica
% mathematical features
\usepackage{amsmath, amssymb}
\usepackage{amsthm}
\theoremstyle{definition}
\usepackage{mathtools}
% screen
\usepackage{ascmac}
% url
\usepackage{url}
% for images
\usepackage{graphicx}
% chemical formula
\usepackage[version=3]{mhchem}
% appendix
\usepackage[toc,page]{appendix}
% add bibliography in TOC
\usepackage[nottoc]{tocbibind}
% hyperlink
\usepackage[svgnames,psnames]{xcolor} % better color for hyperref
\usepackage[colorlinks,citecolor=DarkGreen,linkcolor=FireBrick,linktocpage,unicode]{hyperref} % add hypertext capabilities
% color
\usepackage{color}
% page size and margins
\usepackage{geometry}
% citation
\usepackage{cite}
% Tikz
\usepackage{tikz}
\usetikzlibrary{positioning,arrows.meta,shapes}
% longtable
\usepackage{longtable}

\usepackage{bm}

% *matter for article class
% https://tex.stackexchange.com/questions/154646/is-there-an-easy-way-to-get-the-frontmatter-mainmatter-and-backmatter-in-a-l
\makeatletter
\newcommand\frontmatter{%
  \cleardoublepage
  %\@mainmatterfalse
  \pagenumbering{roman}}
\newcommand\mainmatter{%
  \cleardoublepage
  % \@mainmattertrue
  \pagenumbering{arabic}}
\makeatother

%%%%%%%%%%%%%%%%%%%%%%%%%%%%%%%%%%%%%%%%%%%%%%%%%%%%%%%%%%%%%%%%%%%%%%%%%%%%%%%%
% Macros
%%%%%%%%%%%%%%%%%%%%%%%%%%%%%%%%%%%%%%%%%%%%%%%%%%%%%%%%%%%%%%%%%%%%%%%%%%%%%%%%
% follow https://cdn.journals.aps.org/files/styleguide-pr.pdf
\newcommand{\secref}[1]{Section~\ref{#1}}
\newcommand{\appendixref}[1]{Appendix~\ref{#1}}
\newcommand{\software}[1]{\texttt{#1}}

\newcommand{\term}[1]{\textit{#1}}
\newcommand{\todo}[1]{\textcolor{red}{TODO: #1}}
\newcommand{\norm}[1]{\left\lVert#1\right\rVert}
% https://zrbabbler.hatenablog.com/entry/20120411/1334151482
\newcommand{\relmiddle}[1]{\mathrel{}\middle#1\mathrel{}}
\newcommand{\set}[2]{\left\{ #1 \relmiddle| #2 \right\}}
\newcommand{\Z}{\mathbb{Z}}
\newcommand{\R}{\mathbb{R}}
\DeclareMathOperator*{\argmin}{arg\,min}

\newtheorem{theorem}{Theorem}[section]
\newtheorem{corollary}{Corollary}[theorem]
\newtheorem{lemma}[theorem]{Lemma}
\newtheorem{definition}[theorem]{Definition}

\title{Spin space group}
\author{Kohei Shinohara}
\date{\today}

\begin{document}
\maketitle
\tableofcontents

%%%%%%%%%%%%%%%%%%%%%%%%%%%%%%%%%%%%%%%%%%%%%%%%%%%%%%%%%%%%%%%%%%%%%%%%%%%%%%%
\section{Spin arrangement and spin symmetry operation}

A \term{spin symmetry operation} $(g, \bm{W}) \in \mathrm{E}(3) \times \mathrm{O}(3)$ acts on a pair of position $\bm{r}$ and magnetic moments $\bm{m}$,
\begin{align}
  (g, \bm{W}) (\bm{r}, \bm{m}) \coloneqq (g \bm{r}, \bm{Wm}).
\end{align}
A \term{spin arrangement} is a set of pairs of magnetic moments and a crystal structure.

%%%%%%%%%%%%%%%%%%%%%%%%%%%%%%%%%%%%%%%%%%%%%%%%%%%%%%%%%%%%%%%%%%%%%%%%%%%%%%%
\section{Structure of spin space group}

\subsection{Spin space group}

A spin space group \cite{doi:10.1063/1.1708514,doi:10.1098/rspa.1966.0211,LITVIN1974538,Opechowski1986} is used to classify spin symmetry operations of spin arrangements.

\begin{screen}
  \begin{definition}[Spin space group]
    Let $\mathcal{G}$ be a subgroup of $\mathrm{E}(3) \times \mathrm{O}(3)$.
    When the following $\mathcal{F}(\mathcal{G})$ and $\mathcal{D}(\mathcal{G})$ are space groups, $\mathcal{G}$ is a \term{spin space group} \cite{LITVIN1974538},
    \begin{align}
      \mathcal{F}(\mathcal{G})
        &\coloneqq \set{ g \in \mathrm{E}(3) }{ \exists \bm{W} \in \mathrm{O}(3) \,s.t.\, (g, \bm{W}) \in \mathcal{G} } \\
      \mathcal{D}(\mathcal{G})
        &\coloneqq \set{ g \in \mathrm{E}(3) }{ (g, \bm{E}) \in \mathcal{G} }.
    \end{align}
    For a spin space group $\mathcal{G}$, we call $\mathcal{F}(\mathcal{G})$ a \term{family space group} and $\mathcal{D}(\mathcal{G})$ a \term{maximal space subgroup}.
  \end{definition}
\end{screen}
Reference~\cite{LITVIN1974538} did not impose the condition that $\mathcal{D}(\mathcal{G})$ is crystallographic.
However, this condition seems to be required to guarantee a spin translation group $\mathcal{G}_{\mathrm{st}}(\mathcal{G})$ not to be subperiodic.

\begin{screen}
  \begin{definition}[Spin-space-group type]
    Two spin space groups $\mathcal{G}_{1}$ and $\mathcal{G}_{2}$ belong to the same \term{spin-space-group type} if they are transformed to the other by an orientation-preserving transformation:
    \begin{align}
      \mathcal{G}_{1} \sim \mathcal{G}_{2}
      \overset{\mathrm{def}}{\Longleftrightarrow}
      \exists (\bm{P}, \bm{p}) \,s.t.\, (\bm{P}, \bm{p})^{-1} \mathcal{G}_{1} (\bm{P}, \bm{p}) = \mathcal{G}_{2},
    \end{align}
    where we define the action of transformations to a spin symmetry $(g, \bm{W})$ as
    \begin{align}
      (\bm{P}, \bm{p})^{-1} (g, \bm{W}) (\bm{P}, \bm{p})
      \coloneqq
      ((\bm{P}, \bm{p})^{-1} g (\bm{P}, \bm{p}), \bm{P}^{-1}\bm{W}\bm{P}).
    \end{align}
  \end{definition}
\end{screen}

\subsection{Spin-only group}

\begin{screen}
  \begin{definition}[Spin-only group]
    Let $\mathcal{G}$ be a spin space group.
    A \term{spin-only group} of $\mathcal{G}$ is
    \begin{align}
      \mathcal{P}_{\mathrm{so}}(\mathcal{G})
      \coloneqq
      \set{ \bm{W} \in \mathrm{O}(3) }{ ((\bm{E}, \bm{0}), \bm{W}) \in \mathcal{G} }.
    \end{align}
  \end{definition}
\end{screen}

A spin-only group $\mathcal{P}_{\mathrm{so}}(\mathcal{G})$ is a normal subgroup of $\mathcal{G}$, $1 \times \mathcal{P}_{\mathrm{so}}(\mathcal{G}) \trianglelefteq \mathcal{G}$.
A \term{nontrivial spin space group} of $\mathcal{G}$ is defined as $\mathcal{G} / (1 \times \mathcal{P}_{\mathrm{so}}(\mathcal{G}))$.

When a spin space group $\mathcal{G}$ is a stabilizer of a spin arrangement, spin-only groups are classified as Table~\ref{tab:spin_only_group} \cite{LITVIN1974538,PhysRevX.12.021016}.

\begin{table}[tb]
  \centering
  \caption{Classification of spin-only groups}
  \label{tab:spin_only_group}
  \begin{tabular}{cc}
    \hline \hline
    Spin-only group & Spin arrangement \\
    \hline
    $\infty \infty m \cong \mathrm{O}(3)$ & Nonmagnetic \\
    $\infty m \cong \mathrm{SO}(2) \rtimes \mathbb{Z}_{2} $ & Collinear \\
    $m$ & Coplanar \\
    $1$ & Noncoplaner \\
    \hline \hline
  \end{tabular}
\end{table}

\subsection{Spin translation group}

\begin{screen}
  \begin{definition}[Spin translation group]
    Let $\mathcal{G}$ be a spin space group.
    A \term{spin translation group} of $\mathcal{G}$ is
    \begin{align}
      \mathcal{G}_{\mathrm{st}}(\mathcal{G})
      \coloneqq
      \set{ ((\bm{E}, \bm{v}), \bm{W}) }{ ((\bm{E}, \bm{v}), \bm{W}) \in \mathcal{G}}.
    \end{align}
  \end{definition}
\end{screen}

Reference~\cite{Litvin:a09793} classified the spin translation groups.
A \term{nontrivial spin translation group} of $\mathcal{G}$ is defined as $\mathcal{G}_{\mathrm{st}}(\mathcal{G}) / (1 \times \mathcal{P}_{\mathrm{so}}(\mathcal{G}))$.

We write a translation subgroup and a point group of space group $\mathcal{S}$ as $\mathcal{T}(\mathcal{S})$ and $\mathcal{P}(\mathcal{S})$, respectively,
\begin{align}
  \mathcal{T}(\mathcal{S}) &\coloneqq \set{ (\bm{E}, \bm{t}) }{ (\bm{E}, \bm{t}) \in \mathcal{S} } \\
  \mathcal{P}(\mathcal{S}) &\coloneqq \set{ \bm{W} }{ \exists \bm{v} \, s.t.\, (\bm{W}, \bm{v}) \in \mathcal{S} }.
\end{align}

The spin-only group $\mathcal{P}_{\mathrm{so}}(\mathcal{G})$ is a normal subgroup of $\mathcal{G}_{\mathrm{st}}(\mathcal{G})$.
A translation subgroup of $\mathcal{D}(\mathcal{G})$ is also a normal subgroup of $\mathcal{G}_{\mathrm{st}}(\mathcal{G})$.
Because $(1 \times \mathcal{P}_{\mathrm{so}}(\mathcal{G})) \cap (\mathcal{T}(\mathcal{D}(\mathcal{G})) \times 1) = \{ 1 \}$, we can consider a factor group $\mathcal{G}_{\mathrm{st}}(\mathcal{G}) / ( \mathcal{T}(\mathcal{D}(\mathcal{G})) \times \mathcal{P}_{\mathrm{so}}(\mathcal{G}))$.
We write one of the coset decompositions as
\begin{align}
  \mathcal{G}_{\mathrm{st}}(\mathcal{G})
    &= \bigsqcup_{ \bm{v} }
        ((\bm{E}, \bm{v}), \bm{W}_{\bm{v}})
        \left(
          \mathcal{T}(\mathcal{D}(\mathcal{G}))
          \times
          \mathcal{P}_{\mathrm{so}}(\mathcal{G})
        \right),
\end{align}
where we choose $\bm{W}_{\bm{0}} = \bm{E}$.
Because both $\mathcal{T}(\mathcal{D}(\mathcal{G}))$ and $\mathcal{T}(\mathcal{F}(\mathcal{G}))$ are three-dimensional translation subgroups, a factor group $\mathcal{T}(\mathcal{F}(\mathcal{G})) / \mathcal{T}(\mathcal{D}(\mathcal{G}))$ is finite.
Thus, a set of the coset representatives $\{ ((\bm{E}, \bm{v}), \bm{W}_{\bm{v}}) \}_{\bm{v}}$ is finite.
Be careful that $\bm{W}_{\bm{v}}$ may not be a crystallographic rotation in general!

\subsection{Nontrivial spin point group}

Because $\mathcal{G}_{\mathrm{st}}(\mathcal{G})$ is a normal subgroup of $\mathcal{G}$, we can consider a factor group $\mathcal{G} / \mathcal{G}_{\mathrm{st}}(\mathcal{G})$.
\begin{screen}
  \begin{definition}[Nontrivial spin point group]
    Let $\mathcal{G}$ be a spin space group.
    Consider a factor group $\mathcal{G} / \mathcal{G}_{\mathrm{st}}(\mathcal{G})$ and a corresponding coset decomposition,
    \begin{align}
      \mathcal{G}
        = \bigsqcup_{ \bm{R} \in \mathcal{P}(\mathcal{F}(\mathcal{G})) } ((\bm{R}, \bm{v}_{\bm{R}}), \bm{W}_{\bm{R}}) \mathcal{G}_{\mathrm{st}}(\mathcal{G}),
    \end{align}
    where we choose $\bm{W}_{\bm{E}} = \bm{E}$ and $\bm{v}_{\bm{E}} = \bm{0}$.
    A \term{nontrivial spin point group} of $\mathcal{G}$ is
    \begin{align}
      \mathcal{U}_{\mathrm{np}}(\mathcal{G})
      \coloneqq
      \set{ (\bm{R}, \bm{W}_{\bm{R}}) }{ ((\bm{R}, \bm{v}_{\bm{R}}), \bm{W}_{\bm{R}}) \in \mathcal{G} },
    \end{align}
    which is isomorphic to $\mathcal{G} / \mathcal{G}_{\mathrm{st}}(\mathcal{G})$.
  \end{definition}
\end{screen}

The nontrivial spin point group $\mathcal{U}_{\mathrm{np}}(\mathcal{G})$ is also finite because point group $\mathcal{P}(\mathcal{F}(\mathcal{G}))$ is finite,
Because the map $\bm{W}_{\bullet}: \mathcal{P}(\mathcal{F}(\mathcal{G})) \to \mathrm{O}(3)$ is linear, there exists a normal subgroup $N \trianglelefteq \mathcal{P}(\mathcal{F}(\mathcal{G}))$ such that $\{ \bm{W}_{\bm{R}} \}_{ \bm{R} \in \mathcal{P}(\mathcal{F}(\mathcal{G})) } \cong \mathcal{P}(\mathcal{F}(\mathcal{G})) / N$.
Thus, spin point groups are classified by 32 crystallographic point groups and their normal subgroups \cite{Litvin:a14103}.

%%%%%%%%%%%%%%%%%%%%%%%%%%%%%%%%%%%%%%%%%%%%%%%%%%%%%%%%%%%%%%%%%%%%%%%%%%%%%%%
\section{Spin symmetry operation search}

A spin arrangement is represented by the following four objects:
(1) basis vectors of its lattice $\bm{A} = (\bm{a}_{1}, \bm{a}_{2}, \bm{a}_{3})$,
(2) an array of point coordinates of sites in its unit cell $\bm{X} = (\bm{x}_{1}, \cdots, \bm{x}_{N})$,
(3) an array of atomic types of sites in its unit cell $\bm{T} = (t_{1}, \cdots, t_{N})$,
and (4) an array of magnetic moments of sites in its unit cell $\bm{M} = (\bm{m}_{1}, \cdots, \bm{m}_{N})$,
where $N$ is the number of sites in the unit cell.

First, we consider a crystal structure $(\bm{A}, \bm{X}, \bm{T})$ obtained by ignoring magnetic moments of $(\bm{A}, \bm{X}, \bm{T}, \bm{M})$.
A space group of $(\bm{A}, \bm{X}, \bm{T})$ is written as a stabilizer subgroup of $\mathrm{E}(3)$ that preserves $(\bm{A}, \bm{X}, \bm{T})$:
\begin{align}
    \mathcal{S}
        &\coloneqq \set{
                g = ( \overline{\bm{R}}, \overline{\bm{v}} )_{\bm{A}} \in \mathrm{E}(3)
            }{
                \begin{array}{l}
                    \overline{\bm{R}} \in \mathbb{Z}^{3 \times 3} \\
                    \exists \sigma_{g} \in \mathfrak{S}_{N}, \forall i, \\
                    \overline{\bm{R}} \bm{x}_{i} + \overline{\bm{v}} \equiv \bm{x}_{\sigma_{g}(i)} \, (\mathrm{mod} \, 1) \\
                    t_{i} = t_{\sigma_{g}(i)}
                \end{array}
            },
\end{align}
where $\mathfrak{S}_{N}$ is a symmetric group of degree $N$ and $( \overline{\bm{R}}, \overline{\bm{v}} )_{\bm{A}} \coloneqq ( \bm{A}\overline{\bm{R}}\bm{A}^{-1}, \bm{A}\overline{\bm{v}} )$ \footnote {
  The condition $\overline{\bm{R}} \in \mathbb{Z}^{3 \times 3}$ corresponds to Spglib's convention to omit symmetry operations that do not preserve a supercell.
}.

Let $\bm{A}_{\mathcal{S}}$ be one of primitive basis vectors for $(\bm{A}, \bm{X}, \bm{T})$.
We write a translation subgroup formed by basis vector $\bm{A}$ as
\begin{align}
  \mathcal{T}_{\bm{A}} \coloneqq \set{ (\bm{E}, \bm{n})_{\bm{A}} }{ \bm{n} \in \mathbb{Z}^{3} }.
\end{align}
Then we consider a finite coset decomposition of $\mathcal{S}$,
\begin{align}
  \mathcal{T}_{\bm{A}_{\mathcal{S}}}
    &= \bigsqcup_{ \overline{ \bm{t} }^{\mathcal{S}} }
      \left( \bm{E}, \overline{ \bm{t} }^{\mathcal{S}} \right)_{ \bm{A}_{\mathcal{S}} }
      \mathcal{T}_{\bm{A}} \\
  \mathcal{S}
    &= \bigsqcup_{
          \overline{\bm{R}}^{\mathcal{S}} \in \bm{A}_{\mathcal{S}}^{-1} \mathcal{P}(\mathcal{S}) \bm{A}_{\mathcal{S}}
      }
      \bigsqcup_{
          \left( \bm{E}, \overline{ \bm{t} }^{\mathcal{S}} \right)_{ \bm{A}_{\mathcal{S}} } \mathcal{T}_{\bm{A}}
          \in \mathcal{T}_{\bm{A}_{\mathcal{S}}} / \mathcal{T}_{\bm{A}}
      }
      \left(
        \overline{\bm{R}}^{\mathcal{S}},
        \overline{ \bm{v} }^{\mathcal{S}}(\overline{\bm{R}}^{\mathcal{S}})
      \right)_{ \bm{A}_{\mathcal{S}} }
      \left( \bm{E}, \overline{ \bm{t} }^{\mathcal{S}} \right)_{ \bm{A}_{\mathcal{S}} }
      \mathcal{T}_{\bm{A}},
\end{align}
where $\overline{ \bm{v} }^{\mathcal{S}}(\overline{\bm{R}}^{\mathcal{S}})$ is a translation part for a rotation $\bm{A}_{\mathcal{S}} \overline{\bm{R}}^{\mathcal{S}} \bm{A}_{\mathcal{S}}^{-1}$.
Note that a point group of $\mathcal{S}$ is $\mathcal{P}(\mathcal{S}) = \{ \bm{A}_{\mathcal{S}} \overline{\bm{R}}^{\mathcal{S}} \bm{A}_{\mathcal{S}}^{-1} \}_{ \overline{\bm{R}}^{\mathcal{S}} }$.

A spin space group of $(\bm{A}, \bm{X}, \bm{T}, \bm{M})$ is written as a stabilizer subgroup of $\mathrm{E}(3) \times \mathrm{O}(3)$ that preserves $(\bm{A}, \bm{X}, \bm{T}, \bm{M})$:
\begin{align}
  \mathcal{G}
    &\coloneqq \set{
      (g, \bm{W}) \in \mathrm{E}(3) \times \mathrm{O}(3)
    }{
        \begin{array}{l}
          \exists \sigma_{g} \in \mathfrak{S}_{N}, \forall i, \\
          g \bm{x}_{i} \equiv \bm{x}_{\sigma_{g}(i)} \, (\mathrm{mod} \, 1) \\
          t_{i} = t_{\sigma_{g}(i)} \\
          \bm{W} \bm{m}_{i} = \bm{m}_{\sigma_{g}(i)}
        \end{array}
    } \\
    &= \set{
      (g, \bm{W}) \in \mathcal{S} \times \mathrm{O}(3)
    }{
        \begin{array}{l}
          \exists \sigma_{g} \in \mathfrak{S}_{N}, \forall i, \\
          g \bm{x}_{i} \equiv \bm{x}_{\sigma_{g}(i)} \, (\mathrm{mod} \, 1) \\
          t_{i} = t_{\sigma_{g}(i)} \\
          \bm{W} \bm{m}_{i} = \bm{m}_{\sigma_{g}(i)}
        \end{array}
    } \\
  \mathcal{D} &\coloneqq \mathcal{D}(\mathcal{G}) \\
  \mathcal{F} &\coloneqq \mathcal{F}(\mathcal{G}).
\end{align}

\subsection{Spin-only group search}

\begin{align}
  \mathcal{P}_{\mathrm{so}}
    &\coloneqq \mathcal{P}_{\mathrm{so}}(\mathcal{G})
    = \set{
        \bm{W} \in \mathrm{O}(3)
      }{
        \bm{W} \bm{m}_{i} = \bm{m}_{i} \,(\forall i)
      }
\end{align}

Consider a moment of inertia tensor of $\bm{M}$,
\begin{align}
  \bm{N} \coloneqq \sum_{i=1}^{N} \bm{m}_{i} \otimes \bm{m}_{i}.
\end{align}
Because $\bm{N}$ is a symmetric semi-definite matrix, we can consider its eigen decomposition,
\begin{align}
  \bm{N} = \sum_{r=1}^{3} \sigma_{r} \hat{\bm{n}}_{r} \otimes \hat{\bm{n}}_{r},
\end{align}
where $\sigma_{1} \geq \sigma_{2} \geq \sigma_{3} \geq 0$ and $\{ \hat{\bm{n}}_{r} \}_{r=1}^{3}$ are orthonormal.

\subsubsection{Nonmagnetic spin arrangement}

When all magnetic moments are zero, the spin arrangement is called \term{nonmagnetic}.
A spin-only group of a nonmagnetic spin arrangement is $\mathcal{P}_{\mathrm{so}} = \mathrm{O}(3)$.
In practice, we compare magnetic moments with a tolerance $\epsilon$,
\begin{align}
  \bm{m}_{i} = \bm{0} \rightarrow \norm{ \bm{m}_{i} }_{2} < \epsilon.
\end{align}

\subsubsection{Collinear spin arrangement}

When a spin arrangement is not nonmagnetic and all magnetic moments are parallel or antiparallel, the spin arrangement is called \term{collinear}.
Let $\hat{\bm{n}}$ be a direction parallel or antiparallel to the magnetic moments.
Let $\hat{\bm{n}}'$ be one of the directions perpendicular to the magnetic moments.
A spin-only group of a collinear spin arrangement is
\begin{align*}
  \mathcal{P}_{\mathrm{so}}
    = \set{ R_{\theta \hat{\bm{n}}} m_{\hat{\bm{n}}'}^{l} }{ 0 \leq \theta < 2 \pi, l=0, 1 }.
\end{align*}
Here $R_{\theta \hat{\bm{n}}}$ is a rotation along axis $\hat{\bm{n}}$ by $\theta$.
$m_{\hat{\bm{n}}'}$ is a mirror operation perpendicular to $\hat{\bm{n}}'$.

When a spin arrangement is collinear, the eigenvector $\hat{\bm{n}}_{1}$ should be parallel or antiparallel to all magnetic moments.
We can check it with tolerance as
\begin{align}
  \forall \theta \in [0, 2 \pi ), l \in \{ 0, 1 \}, R_{\theta \hat{\bm{n}}_{1}} m_{\hat{\bm{n}}_{1}'}^{l} \bm{m}_{i} = \bm{m}_{i}
  \rightarrow
  2 \norm{ \bm{m}_{i} - (\bm{m}_{i} \cdot \hat{\bm{n}}_{1})\hat{\bm{n}}_{1} }_{2} < \epsilon.
\end{align}

\subsubsection{Coplanar spin arrangement}

When a spin arrangement is not collinear and all magnetic moments are perpendicular to $\hat{\bm{n}}$, the spin arrangement is called \term{coplanar}.
A spin-only group of a coplanar spin arrangement is $\mathcal{P}_{\mathrm{so}} = \{ 1, m_{\hat{\bm{n}}} \}$.

When a spin arrangement is coplanar, the eigenvector $\hat{\bm{n}}_{3}$ should be perpendicular to all magnetic moments.
We can check it with tolerance as
\begin{align}
  m_{\hat{\bm{n}}_{3}} \bm{m}_{i} = \bm{m}_{i}
  \rightarrow
  2 \norm{ (\bm{m}_{i} \cdot \hat{\bm{n}}_{3}) \hat{\bm{n}}_{3}}_{2} < \epsilon.
\end{align}

\subsubsection{Noncoplanar spin arrangement}

When a spin arrangement is not coplanar, the spin arrangement is called \term{noncoplanar}.
A spin-only group of a noncoplanar spin arrangement is $\mathcal{P}_{\mathrm{so}} = \{ 1 \}$.

\subsection{Translation subgroup of maximal space subgroup}

\begin{figure}
  \centering
  \begin{tikzpicture}
    \coordinate (TA) at (0, 0);
    \coordinate (TAD) at (0, 2);
    \coordinate (TAS) at (0, 4);
    \coordinate (SD) at (2, 3);
    \coordinate (mid) at (2, 5);
    \coordinate (S) at (4, 6);

    \node at (TA) [left] {$\mathcal{T}_{\bm{A}}$};
    \node at (TAD) [left] {$\mathcal{T}_{\bm{A}_{\mathcal{D}}}$};
    \node at (TAS) [left] {$\mathcal{T}_{\bm{A}_{\mathcal{S}}}$};
    \node at (SD) [right] {$\mathcal{S}_{\mathcal{D}}$};
    \node at (S) [right] {$\mathcal{S}$};

    \fill[black] (TA) circle (2pt);
    \fill[black] (TAD) circle (2pt);
    \fill[black] (TAS) circle (2pt);
    \fill[black] (SD) circle (2pt);
    \fill[black] (mid) circle (2pt);
    \fill[black] (S) circle (2pt);

    \draw (TA) -- (TAD);
    \draw (TAD) -- (TAS);
    \draw (TAD) -- (SD);
    \draw (TAS) -- (mid);
    \draw (SD) -- (mid);
    \draw (mid) -- (S);
  \end{tikzpicture}
  \caption{\label{fig:translation_subgroup}Group-subgroup diagram of translation subgroups derived from spin space group.}
\end{figure}

We write the primitive basis vectors of $\mathcal{T}(\mathcal{D})$ as $\bm{A}_{\mathcal{D}}$.
First, we search for $\mathcal{T}(\mathcal{D}) = \mathcal{T}_{\bm{A}_{\mathcal{D}}} = \set{ (\bm{E}, \bm{v}) }{ (\bm{E}, (\bm{E}, \bm{v})) \in \mathcal{G} }$.
The group-subgroup relationships of translation subgroups are shown in Fig.~\ref{fig:translation_subgroup}.
The two basis vectors $\bm{A}$ and $\bm{A}_{\mathcal{S}}$ are related by an integer matrix as
\begin{align}
  \bm{A} = \bm{A}_{\mathcal{S}} \bm{U} \quad (\bm{U} \in \mathbb{Z}^{3 \times 3}).
\end{align}
Since $\mathcal{T}_{\bm{A}_{\mathcal{D}}}$ is a subgroup of $\mathcal{T}_{\bm{A}_{\mathcal{S}}}$, we only need to check finite coset representatives of $\mathcal{T}_{\bm{A}} / \mathcal{T}_{\bm{A}_{\mathcal{S}}}$ \footnote{
  The transformation matrix $\bm{U}$ can be calculated from centerings in $\mathcal{T}_{\bm{A}_{\mathcal{D}}} / \mathcal{T}_{\bm{A}_{\mathcal{S}}}$ and basis vectors of $\bm{A}_{\mathcal{S}}$.
  Primitive basis vectors of $\mathcal{T}_{\bm{A}_{\mathcal{D}}}$ can be taken from three nonzero vectors from the Hermite normal from of a matrix formed by these centerings and basis vectors of $\bm{A}_{\mathcal{S}}$.
}.
We can find translations in $\mathcal{T}_{\bm{A}_{\mathcal{D}}}$ as
\begin{align}
  \mathcal{T}_{\bm{A}_{\mathcal{S}}}
    &= \set{
      \left( \bm{E}, \overline{ \bm{t} }^{\mathcal{S}} \right)_{ \bm{A}_{\mathcal{S}} }
      }{
        \begin{array}{l}
          \exists \sigma \in \mathfrak{S}_{N}, \forall i, \\
          \bm{x}_{i} + \bm{U}^{-1} \overline{ \bm{t} }^{\mathcal{S}} \equiv \bm{x}_{\sigma(i)} \quad (\mathrm{mod}\, 1) \\
          t_{i} = t_{\sigma(i)} \\
          \bm{m}_{i} = \bm{m}_{\sigma(i)}
        \end{array}
      }
      \times \mathcal{T}_{\bm{A}_{\mathcal{D}}}.
\end{align}
Then, we take one of primitive basis vectors $\bm{A}_{\mathcal{D}}$ of with
\begin{align}
  \bm{A}_{\mathcal{D}}
    &= \bm{A}_{\mathcal{S}} \bm{V} \quad (\bm{V} \in \mathbb{Z}^{3 \times 3}).
\end{align}

\subsection{Spin translation group search}

\begin{figure}
  \centering
  \begin{tikzpicture}
    \coordinate (Pso) at (0, 0);
    \coordinate (PsoTAD) at (0, 2);
    \coordinate (PsoTAS) at (0, 4);
    \coordinate (Gst) at (2, 3);
    \coordinate (mid) at (2, 5);
    \coordinate (O3TAS) at (4, 6);

    \node at (Pso) [left] {$\mathcal{P}_{\mathrm{so}}$};
    \node at (PsoTAD) [left] {$\mathcal{P}_{\mathrm{so}} \times \mathcal{T}_{\bm{A}_{\mathcal{D}}}$};
    \node at (PsoTAS) [left] {$\mathcal{P}_{\mathrm{so}} \times \mathcal{T}_{\bm{A}_{\mathcal{S}}}$};
    \node at (Gst) [right] {$\mathcal{G}_{\mathrm{st}}$};
    \node at (O3TAS) [above] {$\mathrm{O}(3) \times \mathcal{T}_{\bm{A}_{\mathcal{S}}}$};

    \fill[black] (Pso) circle (2pt);
    \fill[black] (PsoTAD) circle (2pt);
    \fill[black] (PsoTAS) circle (2pt);
    \fill[black] (Gst) circle (2pt);
    \fill[black] (mid) circle (2pt);
    \fill[black] (O3TAS) circle (2pt);

    \draw (Pso) -- (PsoTAD);
    \draw (PsoTAD) -- (PsoTAS);
    \draw (PsoTAD) -- (Gst);
    \draw (Gst) -- (mid);
    \draw (PsoTAS) -- (mid);
    \draw (mid) -- (O3TAS);
  \end{tikzpicture}
  \caption{\label{fig:spin_translation_group}Group-subgroup diagram of spin translation group.}
\end{figure}

For coset representatives $(\bm{E}, \bm{v}) \mathcal{T}_{\bm{A}_{\mathcal{D}}} \in \mathcal{T}_{\bm{A}_{\mathcal{S}}} / \mathcal{T}_{\bm{A}_{\mathcal{D}}}$, we search for $\bm{W}_{\bm{v}}$ by solving a Procrustes problem as shown in Appendix~\ref{appx:procrustes}.
% Number of generators for $\{ \bm{W}_{\bm{v}} \}$ can be determined from the Smith normal form of $\bm{V}$.
The group-subgroup relationships for the spin translation group $\mathcal{G}_{\mathrm{st}}$ are shown in Fig.~\ref{fig:spin_translation_group}.

\begin{align}
  \mathcal{G}_{\mathrm{st}}
    &=
      \bigsqcup_{ \overline{\bm{v}}^{\mathcal{D}} }
      \bigsqcup_{
        (\bm{E}, \overline{\bm{t}}^{\mathcal{D}})_{\bm{A}_{\mathcal{D}}} \mathcal{T}_{\bm{A}}
          \in \mathcal{T}_{\bm{A}_{\mathcal{D}}} / \mathcal{T}_{\bm{A}}
      }
        ((\bm{E}, \overline{\bm{v}}^{\mathcal{D}})_{\bm{A}_{\mathcal{D}}}, \bm{W}_{ \overline{\bm{v}}^{\mathcal{D}} })
        (\bm{E}, \overline{\bm{t}}^{\mathcal{D}})_{\bm{A}_{\mathcal{D}}}
        \left(
          \mathcal{P}_{\mathrm{so}} \times \mathcal{T}_{\bm{A}}
        \right)
\end{align}

\subsection{Spin space group search}

Some rotations in $\mathcal{P}(\mathcal{S})$ may not be compatible with $\mathcal{T}_{\bm{A}_{\mathcal{D}}}$.
We write the subgroup of $\mathcal{S}$ which is compatible with $\mathcal{T}_{\bm{A}_{\mathcal{D}}}$ as $\mathcal{S}_{\mathcal{D}}$ and its point group as $\mathcal{P}_{\mathcal{D}}$.

\begin{align}
  \mathcal{P}_{\mathcal{D}}
    &\coloneqq \set{
        \bm{A}_{\mathcal{S}} \overline{\bm{R}}^{\mathcal{S}} \bm{A}_{\mathcal{S}}^{-1} \in \mathcal{P}(\mathcal{S})
      }{
        \bm{V}^{-1} \overline{\bm{R}}^{\mathcal{S}} \bm{V} \in \mathbb{Z}^{3 \times 3}
      }
\end{align}

\begin{align}
  \mathcal{S}_{\mathcal{D}}
    &\coloneqq \bigsqcup_{
          \overline{\bm{R}}^{\mathcal{S}} \in \bm{A}_{\mathcal{S}}^{-1} \mathcal{P}_{\mathcal{D}} \bm{A}_{\mathcal{S}}
      } \bigsqcup_{
          \left( \bm{E}, \overline{ \bm{t} }^{\mathcal{S}} \right)_{ \bm{A}_{\mathcal{S}} } \mathcal{T}_{\bm{A}_{\mathcal{D}}}
          \in \mathcal{T}_{\bm{A}_{\mathcal{S}}} / \mathcal{T}_{\bm{A}_{\mathcal{D}}}
      }
      \left(
        \overline{\bm{R}}^{\mathcal{S}},
        \overline{ \bm{v} }^{\mathcal{S}}(\overline{\bm{R}}^{\mathcal{S}})
      \right)_{ \bm{A}_{\mathcal{S}} }
      \left( \bm{E}, \overline{ \bm{t} }^{\mathcal{S}} \right)_{ \bm{A}_{\mathcal{S}} }
      \mathcal{T}_{\bm{A}_{\mathcal{D}}},
\end{align}

Finally, the coset decomposition of $\mathcal{G}$ by $\mathcal{G}_{\mathrm{st}}$ is finite.
The rotation parts of the coset representatives should belong to the point group of $\mathcal{F}(\mathcal{G})$.
The corresponding spin-rotation parts are determined by solving the Procrustes problem as well.
\begin{align}
  \mathcal{G}
    = \bigsqcup_{ \bm{R} \in \mathcal{P}(\mathcal{F}(\mathcal{G})) } ((\bm{R}, \bm{v}_{\bm{R}}), \bm{W}_{\bm{R}}) \mathcal{G}_{\mathrm{st}}
\end{align}

%%%%%%%%%%%%%%%%%%%%%%%%%%%%%%%%%%%%%%%%%%%%%%%%%%%%%%%%%%%%%%%%%%%%%%%%%%%%%%%
\section{Identification of nontrivial spin point group type}

\subsection{Structure of nontrivial spin point group}

It is useful to define the following point groups derived from a nontrivial spin point group $\mathcal{U}_{\mathrm{np}}$.
A \term{family point group} of $\mathcal{U}_{\mathrm{np}}$ is defined as
\begin{align}
  \mathcal{R}(\mathcal{U}_{\mathrm{np}})
    &\coloneqq \set{ \bm{R}_{i} }{ (\bm{R}_{i}, \bm{W}_{i}) \in \mathcal{U}_{\mathrm{np}} }.
\end{align}
A \term{maximal point subgroup} of $\mathcal{U}_{\mathrm{np}}$ is defined as
\begin{align}
  \mathcal{K}(\mathcal{U}_{\mathrm{np}})
    &\coloneqq \set{ \bm{R}_{i} }{ (\bm{R}_{i}, \bm{E}) \in \mathcal{U}_{\mathrm{np}} }.
\end{align}
A \term{family spin point group} of $\mathcal{U}_{\mathrm{np}}$ is defined as
\begin{align}
  \mathcal{B}(\mathcal{U}_{\mathrm{np}})
    &\coloneqq \set{ \bm{W}_{i} }{ (\bm{R}_{i}, \bm{W}_{i}) \in \mathcal{U}_{\mathrm{np}} }.
\end{align}
Note that group operations in $\mathcal{B}(\mathcal{U}_{\mathrm{np}})$ are up to a spin-only group.

\subsection{Nontrivial spin point group type}

Two nontrivial spin point groups, $\mathcal{U}_{\mathrm{np}} = \{ (\bm{R}_{i}, \bm{W}_{i}) \}_{i=1}^{n}$ and $\mathcal{U}_{\mathrm{np}}'= \{ (\bm{R}_{i}', \bm{W}_{i}') \}_{i=1}^{n}$ belong to the same \term{spin point group type} if there exist transformation matrices $\bm{P}, \bm{Q} \in \mathrm{SL}(3, \mathbb{R})$ such that
\begin{align}
  \mathcal{U}_{\mathrm{np}}'
    = \{ (\bm{P}^{-1} \bm{R}_{i} \bm{P}, \bm{Q}^{-1} \bm{W}_{i} \bm{Q}) \}_{i=1}^{n}
    =: (\bm{P}, \bm{Q})^{-1} \mathcal{U}_{\mathrm{np}} (\bm{P}, \bm{Q}).
\end{align}

Representatives of 598 nontrivial spin point groups are tabulated in Ref.~\cite{Litvin:a14103}.
For a given nontrivial spin point group $\mathcal{U}_{\mathrm{np}}$, we write a corresponding representative with the same spin point group type as $\mathcal{U}_{\mathrm{np}}^{\mathrm{std}} = \{ (\bm{R}_{i}^{\mathrm{std}}, \bm{W}_{i}^{\mathrm{std}}) \}_{i=1}^{n}$.

The derived point groups $\mathcal{R}(\mathcal{U}_{\mathrm{np}})$ and $\mathcal{B}(\mathcal{U}_{\mathrm{np}})$ can be transformed into one of 32 representatives of geometric crystal classes.
First, we search for transformation matrices, $\bm{P}$ and $\bm{Q}_{0}$, such that
\begin{align}
  \bm{P}^{-1} \mathcal{R}(\mathcal{U}_{\mathrm{np}}) \bm{P} &= \mathcal{R}(\mathcal{U}_{\mathrm{np}}^{\mathrm{std}}) \\
  \bm{Q}_{0}^{-1} \mathcal{B}(\mathcal{U}_{\mathrm{np}}) \bm{Q}_{0} &= \mathcal{B}(\mathcal{U}_{\mathrm{np}}^{\mathrm{std}}).
\end{align}
The transformation $(\bm{P}, \bm{Q}_{0})$ gives
\begin{align}
  \label{eq:initial_transformed_np}
  (\bm{P}, \bm{Q}_{0})^{-1} \mathcal{U}_{\mathrm{np}} (\bm{P}, \bm{Q}_{0})
    = \{ (\bm{R}_{i}^{\mathrm{std}}, \bm{W}_{\tau(i)}^{\mathrm{std}}) \}_{i=1}^{n},
\end{align}
where $\tau \in \mathfrak{S}_{n}$ is a permutation induced by $\bm{P}$ and $\bm{Q}_{0}$ \footnote{
  Note that permutation $\tau$ is not unique when $\mathcal{K}(\mathcal{U}_{\mathrm{np}})$ is not identity.
  However, we can choose any of the corresponding permutations for the later step.
}.

We consider an automorphism group of $\mathcal{B}(\mathcal{U}_{\mathrm{np}}^{\mathrm{std}})$ to transform the nontrivial spin point group in Eq.~\eqref{eq:initial_transformed_np} to $\mathcal{B}(\mathcal{U}_{\mathrm{np}}^{\mathrm{std}})$.
The automorphism group of a point group $\mathcal{P} = \{ \bm{R}_{i} \}_{i=1}^{n}$ is a permutation group induced by its affine normalizer \cite{Gubler1982},
\begin{align}
  \mathrm{Aut}(\mathcal{P})
    \coloneqq \set {\tau \in \mathfrak{S}_{n} }{ \exists \bm{Q} \in \mathcal{N}_{\mathrm{A}(3)}(\mathcal{P}), \bm{Q}^{-1}\bm{R}_{i}\bm{Q} = \bm{R}_{\tau(i)} \, (\forall i) }.
\end{align}
The automorphism group is isomorphic to a factor group of $\mathcal{N}_{\mathrm{A}(3)}(\mathcal{P}) / \mathcal{C}_{\mathrm{A}(3)}(\mathcal{P})$, where $\mathcal{C}_{\mathrm{A}(3)}(\mathcal{P})$ is a centralizer of $\mathcal{P}$,
\begin{align}
  \mathcal{C}_{\mathrm{A}(3)}(\mathcal{P})
    &\coloneqq \set{ \bm{Q} \in \mathrm{A}(3) }{ \bm{Q}^{-1} \bm{R}_{i} \bm{Q} = \bm{R}_{i} \, (\forall i) }.
\end{align}
The automorphism groups for crystallographic point groups are tabulated in Table~\ref{tab:normalizer_centralizer_pg}, which is based on Table 3.5.4.2 of Ref.~\cite{koch2016normalizers}.

\begin{table}[tb]
  \centering
  \caption{Normalizers and centralizers of crystallographic point groups}
  \label{tab:normalizer_centralizer_pg}
  \begin{tabular}{cccc}
    \hline \hline
    Point groups $\mathcal{P}$
      & $\mathcal{N}_{\mathrm{A}(3)}(\mathcal{P})$
      & $\mathcal{C}_{\mathrm{A}(3)}(\mathcal{P})$
      & $ | \mathrm{Aut}(\mathcal{P}) | $ \\
    \hline
    $1$, $\overline{1}$ & $\infty\infty m$ & $\infty\infty m$ & $1$ \\
    $2$, $m$, $2/m$, $4$, $\overline{4}$, $4/m$, $3$, $\overline{3}$, $6$, $\overline{6}$, $6/m$
      & $\infty / m m$ & $\infty / m m$ & $1$ \\
    $222$, $mmm$ & $m\overline{3}m$ & $mmm$ & 6 \\
    $mm2$ & $4/mmm$ & $mmm$ & 2 \\ % \{ 1, 4^{+} \}
    $\overline{4}2m$ & $4/mmm$ & $2/m$ & 4 \\
    $422$, $4mm$, $4/mmm$ & $8/mmm$ & $2/m$ & 8 \\
    $32$, $3m$, $\overline{3}m$, $\overline{6}2m$ & $6/mmm$ & $2/m$ & 6 \\
    $622$, $6mm$, $6/mmm$ & $12/mmm$ & $2/m$ & 12 \\
    $23$, $m\overline{3}$, $432$, $\overline{4}3m$, $m\overline{3}m$
      & $m\overline{3}m$ & $\overline{1}$ & 24 \\ % 432
    \hline \hline
  \end{tabular}
\end{table}

Then, we try to search for $\bm{Q}' \in \mathrm{Aut}_{\mathrm{A}(3)}(\mathcal{B}(\mathcal{U}_{\mathrm{np}}^{\mathrm{std}}))$ such that
\begin{align}
  \bm{Q}'^{-1} \bm{W}_{i} \bm{Q}' = \bm{W}_{\tau^{-1}(i)}
    \quad (i = 1, \cdots, n).
\end{align}
If such a transformation $\bm{Q}'$ exists, an adjusted transformation $(\bm{P}, \bm{Q}_{0}\bm{Q}')$ gives
\begin{align}
  (\bm{P}, \bm{Q}_{0}\bm{Q}')^{-1} \mathcal{U}_{\mathrm{np}} (\bm{P}, \bm{Q}_{0}\bm{Q}') = \mathcal{U}_{\mathrm{np}}^{\mathrm{std}}.
\end{align}


% Let $\mathcal{P}_{\mathrm{c}}$ be one of 32 representatives of geometric crystal classes (Table 3.2.3.1 of ITA).
% Then, we consider a normal subgroup $N \trianglelefteq \mathcal{P}_{\mathrm{c}}$.
% Let $\mathcal{B}_{\mathrm{c}} = \{ \bm{W}_{\bm{R}} \}_{\bm{R}} \cong \mathcal{P}_{\mathrm{c}} / N$ be one of 32 representatives of geometric crystal classes.
% Let $f: \mathcal{P}_{\mathrm{c}} / N \to \mathcal{B}_{\mathrm{c}}$ be an isomorphism.
%
% To enumerate all spin point group types, it is sufficient to enumerate $(\mathcal{P}_{\mathrm{c}}, N, \mathcal{B}_{\mathrm{c}}, f)$.
% \begin{align}
%   \bm{W}_{\bm{R}} &= \bm{E} \quad (\bm{R} \in N) \\
%   \bm{W}_{\bm{R}} &= f(\bm{R}_{i}) \quad (\bm{R} \in \bm{R}_{i} N )
% \end{align}

%%%%%%%%%%%%%%%%%%%%%%%%%%%%%%%%%%%%%%%%%%%%%%%%%%%%%%%%%%%%%%%%%%%%%%%%%%%%%%%
\section{Symmetrization of spin arrangement}

\todo{}

%%%%%%%%%%%%%%%%%%%%%%%%%%%%%%%%%%%%%%%%%%%%%%%%%%%%%%%%%%%%%%%%%%%%%%%%%%%%%%%
\appendix

\section{\label{appx:procrustes}Procrustes problem}

For $g \in \mathcal{S}$, we consider to search for $\bm{W} \in \mathrm{O}(3)$ such that $\bm{W} \bm{m}_{i} = \bm{m}_{\sigma_{g}(i)}$ for all $i$.
We choose a candidate $\tilde{\bm{W}}$ by solving the following Procrustes problem \cite{10.1093/acprof:oso/9780198510581.001.0001}:
\begin{align}
  \tilde{\bm{W}}
    &=\argmin_{ \bm{W} \in \mathrm{O}(3) } \norm{ \bm{M}_{g} - \bm{W} \bm{M} }_{F} \\
  \bm{M}_{g}
    &\coloneqq \left( \bm{m}_{\sigma_{g}(1)} \cdots \bm{m}_{\sigma_{g}(N)} \right)
    \quad \in \mathbb{R}^{3 \times N}.
\end{align}
We can write $\tilde{\bm{W}}$ with singular value decomposition of $\bm{M}_{g} \bm{M}^{\top}$,
\begin{align}
  \bm{M}_{g} \bm{M}^{\top}
    = \bm{U} \bm{\Sigma} \bm{V}^{\top}
\end{align}
as
\begin{align}
  \tilde{\bm{W}} = \bm{U} \bm{V}^{\top}.
\end{align}

\section{\label{appx:group}Group product}

\begin{screen}
  \begin{definition}[Internal direct product]
    Let $G$ be a group.
    If $G$ has normal subgroups $H, K \trianglelefteq G$ such that $HK = G$ and $H \cap K = \{ 1 \}$, $G$ is called a \term{internal direct product} of $H$ and $K$, denoted as $G = H \times K$.
  \end{definition}
\end{screen}

\begin{screen}
  \begin{definition}[Internal semidirect product]
    Let $G$ be a group.
    If $G$ has a normal subgroup $N \trianglelefteq G$ and a subgroup $H \leq G$ such that $NH = G$ and $N \cap H = \{ 1 \}$, $G$ is called a \term{internal semidirect product} of $N$ and $H$, denoted as $G = N \rtimes H = H \ltimes N$.
  \end{definition}
\end{screen}

If group $G = N \rtimes H$ is a semidirect product, there is a group homomorphism $\phi: H \to \mathrm{Aut}(N)$ with $\phi(h)(n) = h n h^{-1}$ for $h \in H$ and $n \in N$.
We also write the semidirect product as $G = N \rtimes_{\phi} H = H \ltimes_{\phi} N$.

A group is not necessarily a semidirect product of its nontrivial normal subgroup and a subgroup.
The cyclic group $\mathbb{Z}_{4}$ is a counterexample.
$\mathbb{Z}_{4}$ has an exact sequence $\mathbb{Z}_{2} \overset{\iota}{\hookrightarrow} \mathbb{Z}_{4} \overset{\times 2}{\twoheadrightarrow} \mathbb{Z}_{2}$, where $\iota: \mathbb{Z}_{2} \ni n \mapsto n \in \mathbb{Z}_{4}$ and $\times 2: \mathbb{Z}_{4} \ni n \mapsto 2n \in \mathbb{Z}_{2}$.
However, if $\mathbb{Z}_{4}$ was a semidirect product $\mathbb{Z}_{2} \rtimes \mathbb{Z}_{2}$, it would be isomorphic to $\mathbb{Z}_{2} \times \mathbb{Z}_{2}$ because $\mathrm{Aut}(\mathbb{Z}_{2}) = \{ 1 \}$.

Let $\mathcal{P}_{\mathrm{so}}$ be a spin-only group.
Let $\mathcal{B}$ be a subgroup of a normalizer of $\mathcal{P}_{\mathrm{so}}$ in $\mathrm{O}(3)$.
Then, we prove $\mathcal{B}$ is an internal direct product of $\mathcal{P}_{\mathrm{so}}$ and a normal subgroup of $\mathcal{B}$ as follows.

The normalizers of spin-only groups are classified in Table~\ref{tab:spin_only_normalizers}.
For the noncoplanar case, $\mathcal{B} = 1 \times \mathcal{B} = \mathcal{P}_{\mathrm{so}} \times \mathcal{B}$ is trivially an internal direct product.
For the collinear case, $\mathcal{B}$ is $\mathcal{P}_{\mathrm{so}}$ or $\mathcal{P}_{\mathrm{so}} \times \overline{1}$, which are both internal direct products.
For the coplanar case, $\mathcal{B}$ should be one of the following (see Tables 3.2.1.3 and 3.2.1.6 of Ref.~\cite{hahn2016point}),
\begin{itemize}
  \item $C_{1h}$
  \item $C_{nh} = C_{1h} \times C_{n}$ ($n \geq 2$, $n$ is even)
  \item $D_{nh} = C_{1h} \times D_{n}$ ($n \geq 2$)
  % C_1h is not normal in T_h, O_h, and I_h.
  \item $C_{\infty h} = C_{1h} \times C_{\infty}$
  \item $D_{\infty h} = C_{1h} \times D_{\infty}$.
\end{itemize}
Thus, $\mathcal{B}$ is an internal direct product.
For the nonmagnetic case, $\mathcal{B} = \mathcal{P}_{\mathrm{so}}$ is trivially an internal direct product.

\begin{table}[tb]
  \centering
  \caption{
    Normalizers of spin-only groups based on Table~3.5.4.2 of Ref.~\cite{koch2016normalizers}.
  }
  \label{tab:spin_only_normalizers}
  \begin{tabular}{ccc}
    \hline \hline
    Spin arrangement & Spin-only group $\mathcal{P}_{\mathrm{so}}$ & Normalizer \\
    \hline
    Nonmagnetic & $\infty \infty m \,(\mathrm{O}(3))$ & $\infty \infty m \,(\mathrm{O}(3))$ \\
    Collinear & $\infty m \,(C_{\infty v})$ & $\infty / mm \, (D_{\infty h})$ \\
    Coplanar & $m \,(C_{1h})$ & $\infty / mm \, (D_{\infty h})$ \\
    Noncoplanar & $1 \, (C_{1})$ & $\infty \infty m \,(\mathrm{O}(3))$ \\
    \hline \hline
  \end{tabular}
\end{table}

\begin{table}[tb]
  \centering
  \caption{
    General point groups in three dimensions, based on Tables 3.2.1.6 of Ref.~\cite{hahn2016point}
  }
  \label{tab:point_groups}
  \begin{tabular}{cccc}
    \hline \hline
    & Schoenflies & Hellmann-Mauguin & Order \\
    \hline
    % Involutional
    & $C_{1}$ & $1$                     & $1$ \\
    & $C_{2}$ & $2$                     & $2$ \\
    & $C_{1h}$ & $m \equiv \overline{2}$ & $2$ \\
    & $C_{i}$ & $\overline{1}$          & $2$ \\
    \hline
    % Cyclic
    $n \geq 3$                     & $C_{n}$               & $n$            & $n$ \\
    $n \geq 2$, \mbox{$n$ is even} & $S_{n}$               & $\overline{n}$ & $n$ \\
    $n \geq 2$, \mbox{$n$ is even} & $C_{nh}$              & $n / m$        & $2n$ \\
    $n \geq 3$, \mbox{$n$ is odd}  & $C_{nh} \equiv S_{n}$ & $\overline{n}$ & $2n$ \\
    $n \geq 2$, \mbox{$n$ is even} & $C_{nv}$              & $nmm$          & $2n$ \\
    $n \geq 3$, \mbox{$n$ is odd}  & $C_{nv}$              & $nm$           & $2n$ \\
    \hline
    % Dihedral
    $n \geq 2$, \mbox{$n$ is even} & $D_{n}$  & $n22$             & $2n$ \\
    $n \geq 3$, \mbox{$n$ is odd}  & $D_{n}$  & $n2$              & $2n$ \\
    $n \geq 2$, \mbox{$n$ is even} & $D_{nd}$ & $\overline{n}2m$  & $4n$ \\
    $n \geq 3$, \mbox{$n$ is odd}  & $D_{nd}$ & $\overline{n}m$   & $4n$ \\
    $n \geq 2$, \mbox{$n$ is even} & $D_{nh}$ & $n/mmm$           & $4n$ \\
    $n \geq 3$, \mbox{$n$ is odd}  & $D_{nh}$ & $\overline{2n}2m$ & $4n$ \\
    \hline
    % Polyhedral
    & $T$     & $23$             & $12$ \\
    & $T_{h}$ & $m\overline{3}$  & $24$ \\
    & $T_{d}$ & $\overline{4}3m$ & $24$ \\
    & $O$     & $432$            & $24$ \\
    & $O_{h}$ & $m\overline{3}m$ & $48$ \\
    & $I$     & $235$            & $60$ \\
    & $I_{h}$ & $m\overline{35}$ & $120$ \\
    \hline
    % Continuous
    & $C_{\infty}$                       & $\infty$          & $\infty$ \\
    & $C_{\infty h} \equiv S_{\infty}$   & $\infty / m$      & $\infty$ \\
    & $C_{\infty v}$                     & $\infty m$        & $\infty$ \\
    & $D_{\infty}$                       & $\infty 2$        & $\infty$ \\
    & $D_{\infty h} \equiv D_{\infty d}$ & $\infty / mm$     & $\infty$\\
    & $K$                                & $\infty \infty$   & $\infty$\\
    & $K_{h}$                            & $\infty \infty m$ & $\infty$\\
    \hline \hline
  \end{tabular}
\end{table}

Ref.~\cite{RevModPhys.35.641}

\section{\label{appx:spin_point_group_table}Tabulation of nontrivial spin point group types}

\newpage

\begin{longtable}{ccccc}
  \caption{Nontrivial spin point group types}
  \label{tab:spin_point_group_types} \\
  \hline \hline
  Litvin number & $R$ & $r$ & $B$ & Symbol \\
  \hline
  \endfirsthead
  \multicolumn{5}{c}{\tablename\ \thetable\ (\textit{cont.})} \\
  Litvin number & $R$ & $r$ & $B$ & Symbol \\
  \hline
  \endhead
  \endfoot
  \endlastfoot
  1 & $1$ & $1$ & $1$ & ${}^{1} 1 $\\
  2 & $\overline{1}$ & $\overline{1}$ & $1$ & ${}^{1} \overline{1} $\\
  3 &  & $1$ & $2$ & ${}^{2} \overline{1} $\\
  4 &  &  & $m$ & ${}^{m} \overline{1} $\\
  5 &  &  & $\overline{1}$ & ${}^{\overline{1}} \overline{1} $\\
  6 & $2$ & $2$ & $1$ & ${}^{1} 2 $\\
  7 &  & $1$ & $2$ & ${}^{2} 2 $\\
  8 &  &  & $m$ & ${}^{m} 2 $\\
  9 &  &  & $\overline{1}$ & ${}^{\overline{1}} 2 $\\
  10 & $m$ & $m$ & $1$ & ${}^{1} m $\\
  11 &  & $1$ & $2$ & ${}^{2} m $\\
  12 &  &  & $m$ & ${}^{m} m $\\
  13 &  &  & $\overline{1}$ & ${}^{\overline{1}} m $\\
  14 & $2/m$ & $2/m$ & $1$ & ${}^{1} 2  / {}^{1} m $\\
  15 &  & $2$ & $2$ & ${}^{1} 2  / {}^{2} m $\\
  16 &  &  & $m$ & ${}^{1} 2  / {}^{m} m $\\
  17 &  &  & $\overline{1}$ & ${}^{1} 2  / {}^{\overline{1}} m $\\
  18 &  & $m$ & $2$ & ${}^{2} 2  / {}^{1} m $\\
  19 &  &  & $m$ & ${}^{m} 2  / {}^{1} m $\\
  20 &  &  & $\overline{1}$ & ${}^{\overline{1}} 2  / {}^{1} m $\\
  21 &  & $\overline{1}$ & $2$ & ${}^{2} 2  / {}^{2} m $\\
  22 &  &  & $m$ & ${}^{m} 2  / {}^{m} m $\\
  23 &  &  & $\overline{1}$ & ${}^{\overline{1}} 2  / {}^{\overline{1}} m $\\
  24 &  & $1$ & $222$ & ${}^{2_{100}} 2  / {}^{2_{010}} m $\\
  25 &  &  & $mm2$ & ${}^{2} 2  / {}^{m_{100}} m $\\
  26 &  &  &  & ${}^{m_{100}} 2  / {}^{m_{010}} m $\\
  27 &  &  &  & ${}^{m_{100}} 2  / {}^{2} m $\\
  28 &  &  & $2/m$ & ${}^{2} 2  / {}^{m} m $\\
  29 &  &  &  & ${}^{2} 2  / {}^{\overline{1}} m $\\
  30 &  &  &  & ${}^{\overline{1}} 2  / {}^{m} m $\\
  31 &  &  &  & ${}^{\overline{1}} 2  / {}^{2} m $\\
  32 &  &  &  & ${}^{m} 2  / {}^{\overline{1}} m $\\
  33 &  &  &  & ${}^{m} 2  / {}^{2} m $\\
  34 & $mm2$ & $mm2$ & $1$ & ${}^{1} m {}^{1} m {}^{1} 2 $\\
  35 &  & $2$ & $2$ & ${}^{2} m {}^{2} m {}^{1} 2 $\\
  36 &  &  & $m$ & ${}^{m} m {}^{m} m {}^{1} 2 $\\
  37 &  &  & $\overline{1}$ & ${}^{\overline{1}} m {}^{\overline{1}} m {}^{1} 2 $\\
  38 &  & $m$ & $2$ & ${}^{1} m {}^{2} m {}^{2} 2 $\\
  39 &  &  & $m$ & ${}^{1} m {}^{m} m {}^{m} 2 $\\
  40 &  &  & $\overline{1}$ & ${}^{1} m {}^{\overline{1}} m {}^{\overline{1}} 2 $\\
  41 &  & $1$ & $222$ & ${}^{2_{100}} m {}^{2_{010}} m {}^{2_{001}} 2 $\\
  42 &  &  & $mm2$ & ${}^{m_{100}} m {}^{m_{010}} m {}^{2} 2 $\\
  43 &  &  &  & ${}^{m_{100}} m {}^{2} m {}^{m_{010}} 2 $\\
  44 &  &  & $2/m$ & ${}^{m} m {}^{\overline{1}} m {}^{2} 2 $\\
  45 &  &  &  & ${}^{m} m {}^{2} m {}^{\overline{1}} 2 $\\
  46 &  &  &  & ${}^{\overline{1}} m {}^{2} m {}^{m} 2 $\\
  47 & $222$ & $222$ & $1$ & ${}^{1} 2 {}^{1} 2 {}^{1} 2 $\\
  48 &  & $2$ & $2$ & ${}^{1} 2 {}^{2} 2 {}^{2} 2 $\\
  49 &  &  & $m$ & ${}^{1} 2 {}^{m} 2 {}^{m} 2 $\\
  50 &  &  & $\overline{1}$ & ${}^{1} 2 {}^{\overline{1}} 2 {}^{\overline{1}} 2 $\\
  51 &  & $1$ & $222$ & ${}^{2_{100}} 2 {}^{2_{010}} 2 {}^{2_{001}} 2 $\\
  52 &  &  & $mm2$ & ${}^{m_{100}} 2 {}^{m_{010}} 2 {}^{2} 2 $\\
  53 &  &  & $2/m$ & ${}^{2} 2 {}^{\overline{1}} 2 {}^{m} 2 $\\
  54 & $mmm$ & $mmm$ & $1$ & ${}^{1} m {}^{1} m {}^{1} m $\\
  55 &  & $2/m$ & $2$ & ${}^{1} m {}^{2} m {}^{2} m $\\
  56 &  &  & $m$ & ${}^{1} m {}^{m} m {}^{m} m $\\
  57 &  &  & $\overline{1}$ & ${}^{1} m {}^{\overline{1}} m {}^{\overline{1}} m $\\
  58 &  & $mm2$ & $2$ & ${}^{1} m {}^{1} m {}^{2} m $\\
  59 &  &  & $m$ & ${}^{1} m {}^{1} m {}^{m} m $\\
  60 &  &  & $\overline{1}$ & ${}^{1} m {}^{1} m {}^{\overline{1}} m $\\
  61 &  & $222$ & $2$ & ${}^{2} m {}^{2} m {}^{2} m $\\
  62 &  &  & $m$ & ${}^{m} m {}^{m} m {}^{m} m $\\
  63 &  &  & $\overline{1}$ & ${}^{\overline{1}} m {}^{\overline{1}} m {}^{\overline{1}} m $\\
  64 &  & $2$ & $222$ & ${}^{2_{001}} m {}^{2_{001}} m {}^{2_{100}} m $\\
  65 &  &  & $mm2$ & ${}^{m_{100}} m {}^{m_{100}} m {}^{m_{010}} m $\\
  66 &  &  &  & ${}^{m_{100}} m {}^{m_{100}} m {}^{2} m $\\
  67 &  &  &  & ${}^{2} m {}^{2} m {}^{m_{100}} m $\\
  68 &  &  & $2/m$ & ${}^{m} m {}^{m} m {}^{\overline{1}} m $\\
  69 &  &  &  & ${}^{m} m {}^{m} m {}^{2} m $\\
  70 &  &  &  & ${}^{2} m {}^{2} m {}^{\overline{1}} m $\\
  71 &  &  &  & ${}^{2} m {}^{2} m {}^{m} m $\\
  72 &  &  &  & ${}^{\overline{1}} m {}^{\overline{1}} m {}^{2} m $\\
  73 &  &  &  & ${}^{\overline{1}} m {}^{\overline{1}} m {}^{m} m $\\
  74 &  & $m$ & $222$ & ${}^{2_{100}} m {}^{2_{010}} m {}^{1} m $\\
  75 &  &  & $mm2$ & ${}^{m_{100}} m {}^{m_{010}} m {}^{1} m $\\
  76 &  &  &  & ${}^{m_{010}} m {}^{2} m {}^{1} m $\\
  77 &  &  & $2/m$ & ${}^{m} m {}^{\overline{1}} m {}^{1} m $\\
  78 &  &  &  & ${}^{m} m {}^{2} m {}^{1} m $\\
  79 &  &  &  & ${}^{\overline{1}} m {}^{2} m {}^{1} m $\\
  80 &  & $\overline{1}$ & $222$ & ${}^{2_{100}} m {}^{2_{010}} m {}^{2_{001}} m $\\
  81 &  &  & $mm2$ & ${}^{m_{100}} m {}^{m_{010}} m {}^{2} m $\\
  82 &  &  & $2/m$ & ${}^{2} m {}^{\overline{1}} m {}^{m} m $\\
  83 &  & $1$ & $mmm$ & ${}^{m_{100}} m {}^{m_{010}} m {}^{m_{001}} m $\\
  84 &  &  &  & ${}^{m_{100}} m {}^{2_{010}} m {}^{2_{001}} m $\\
  85 &  &  &  & ${}^{m_{001}} m {}^{2_{100}} m {}^{2_{001}} m $\\
  86 &  &  &  & ${}^{2_{001}} m {}^{m_{100}} m {}^{m_{001}} m $\\
  87 &  &  &  & ${}^{2_{001}} m {}^{\overline{1}} m {}^{m_{100}} m $\\
  88 &  &  &  & ${}^{2_{001}} m {}^{\overline{1}} m {}^{2_{100}} m $\\
  89 &  &  &  & ${}^{\overline{1}} m {}^{m_{001}} m {}^{m_{100}} m $\\
  90 & $4$ & $4$ & $1$ & ${}^{1} 4 $\\
  91 &  & $2$ & $2$ & ${}^{2} 4 $\\
  92 &  &  & $m$ & ${}^{m} 4 $\\
  93 &  &  & $\overline{1}$ & ${}^{\overline{1}} 4 $\\
  94 &  & $1$ & $4$ & ${}^{4^{+}} 4 $\\
  95 &  &  & $\overline{4}$ & ${}^{\overline{4}^{+}} 4 $\\
  96 & $\overline{4}$ & $\overline{4}$ & $1$ & ${}^{1} \overline{4} $\\
  97 &  & $2$ & $2$ & ${}^{2} \overline{4} $\\
  98 &  &  & $m$ & ${}^{m} \overline{4} $\\
  99 &  &  & $\overline{1}$ & ${}^{\overline{1}} \overline{4} $\\
  100 &  & $1$ & $4$ & ${}^{4^{+}} \overline{4} $\\
  101 &  &  & $\overline{4}$ & ${}^{\overline{4}^{+}} \overline{4} $\\
  102 & $4/m$ & $4/m$ & $1$ & ${}^{1} 4  / {}^{1} m $\\
  103 &  & $2/m$ & $2$ & ${}^{2} 4  / {}^{1} m $\\
  104 &  &  & $m$ & ${}^{m} 4  / {}^{1} m $\\
  105 &  &  & $\overline{1}$ & ${}^{\overline{1}} 4  / {}^{1} m $\\
  106 &  & $\overline{4}$ & $2$ & ${}^{2} 4  / {}^{2} m $\\
  107 &  &  & $m$ & ${}^{m} 4  / {}^{m} m $\\
  108 &  &  & $\overline{1}$ & ${}^{\overline{1}} 4  / {}^{\overline{1}} m $\\
  109 &  & $4$ & $2$ & ${}^{1} 4  / {}^{2} m $\\
  110 &  &  & $m$ & ${}^{1} 4  / {}^{m} m $\\
  111 &  &  & $\overline{1}$ & ${}^{1} 4  / {}^{\overline{1}} m $\\
  112 &  & $\overline{1}$ & $4$ & ${}^{4^{+}} 4  / {}^{2} m $\\
  113 &  &  & $\overline{4}$ & ${}^{\overline{4}^{+}} 4  / {}^{2} m $\\
  114 &  & $m$ & $4$ & ${}^{4^{+}} 4  / {}^{1} m $\\
  115 &  &  & $\overline{4}$ & ${}^{\overline{4}^{+}} 4  / {}^{1} m $\\
  116 &  & $2$ & $222$ & ${}^{2_{001}} 4  / {}^{2_{100}} m $\\
  117 &  &  & $mm2$ & ${}^{2} 4  / {}^{m_{100}} m $\\
  118 &  &  &  & ${}^{m_{100}} 4  / {}^{2} m $\\
  119 &  &  &  & ${}^{m_{100}} 4  / {}^{m_{010}} m $\\
  120 &  &  & $2/m$ & ${}^{2} 4  / {}^{\overline{1}} m $\\
  121 &  &  &  & ${}^{2} 4  / {}^{m} m $\\
  122 &  &  &  & ${}^{\overline{1}} 4  / {}^{2} m $\\
  123 &  &  &  & ${}^{\overline{1}} 4  / {}^{m} m $\\
  124 &  &  &  & ${}^{m} 4  / {}^{2} m $\\
  125 &  &  &  & ${}^{m} 4  / {}^{\overline{1}} m $\\
  126 &  & $1$ & $4/m$ & ${}^{4^{+}} 4  / {}^{m} m $\\
  127 &  &  &  & ${}^{4^{+}} 4  / {}^{\overline{1}} m $\\
  128 &  &  &  & ${}^{\overline{4}^{+}} 4  / {}^{m} m $\\
  129 &  &  &  & ${}^{\overline{4}^{+}} 4  / {}^{\overline{1}} m $\\
  130 & $422$ & $422$ & $1$ & ${}^{1} 4 {}^{1} 2 {}^{1} 2 $\\
  131 &  & $4$ & $2$ & ${}^{1} 4 {}^{2} 2 {}^{2} 2 $\\
  132 &  &  & $m$ & ${}^{1} 4 {}^{m} 2 {}^{m} 2 $\\
  133 &  &  & $\overline{1}$ & ${}^{1} 4 {}^{\overline{1}} 2 {}^{\overline{1}} 2 $\\
  134 &  & $222$ & $2$ & ${}^{2} 4 {}^{1} 2 {}^{2} 2 $\\
  135 &  &  & $m$ & ${}^{m} 4 {}^{1} 2 {}^{m} 2 $\\
  136 &  &  & $\overline{1}$ & ${}^{\overline{1}} 4 {}^{1} 2 {}^{\overline{1}} 2 $\\
  137 &  & $2$ & $222$ & ${}^{2_{100}} 4 {}^{2_{010}} 2 {}^{2_{001}} 2 $\\
  138 &  &  & $mm2$ & ${}^{2} 4 {}^{m_{100}} 2 {}^{m_{010}} 2 $\\
  139 &  &  &  & ${}^{m_{100}} 4 {}^{2} 2 {}^{m_{010}} 2 $\\
  140 &  &  & $2/m$ & ${}^{2} 4 {}^{\overline{1}} 2 {}^{m} 2 $\\
  141 &  &  &  & ${}^{\overline{1}} 4 {}^{2} 2 {}^{m} 2 $\\
  142 &  &  &  & ${}^{m} 4 {}^{\overline{1}} 2 {}^{2} 2 $\\
  143 &  & $1$ & $422$ & ${}^{4^{+}} 4 {}^{2_{100}} 2 {}^{2_{1\overline{1}0}} 2 $\\
  144 &  &  & $4mm$ & ${}^{4^{+}} 4 {}^{m_{100}} 2 {}^{m_{1\overline{1}0}} 2 $\\
  145 &  &  & $\overline{4}2m$ & ${}^{\overline{4}^{+}} 4 {}^{2_{100}} 2 {}^{m_{1\overline{1}0}} 2 $\\
  146 & $4mm$ & $4mm$ & $1$ & ${}^{1} 4 {}^{1} m {}^{1} m $\\
  147 &  & $4$ & $2$ & ${}^{1} 4 {}^{2} m {}^{2} m $\\
  148 &  &  & $m$ & ${}^{1} 4 {}^{m} m {}^{m} m $\\
  149 &  &  & $\overline{1}$ & ${}^{1} 4 {}^{\overline{1}} m {}^{\overline{1}} m $\\
  150 &  & $mm2$ & $2$ & ${}^{2} 4 {}^{1} m {}^{2} m $\\
  151 &  &  & $m$ & ${}^{m} 4 {}^{1} m {}^{m} m $\\
  152 &  &  & $\overline{1}$ & ${}^{\overline{1}} 4 {}^{1} m {}^{\overline{1}} m $\\
  153 &  & $2$ & $222$ & ${}^{2_{100}} 4 {}^{2_{010}} m {}^{2_{001}} m $\\
  154 &  &  & $mm2$ & ${}^{2} 4 {}^{m_{100}} m {}^{m_{010}} m $\\
  155 &  &  &  & ${}^{m_{100}} 4 {}^{2} m {}^{m_{010}} m $\\
  156 &  &  & $2/m$ & ${}^{2} 4 {}^{\overline{1}} m {}^{m} m $\\
  157 &  &  &  & ${}^{\overline{1}} 4 {}^{2} m {}^{m} m $\\
  158 &  &  &  & ${}^{m} 4 {}^{2} m {}^{\overline{1}} m $\\
  159 &  & $1$ & $422$ & ${}^{4^{+}} 4 {}^{2_{100}} m {}^{2_{1\overline{1}0}} m $\\
  160 &  &  & $4mm$ & ${}^{4^{+}} 4 {}^{m_{100}} m {}^{m_{1\overline{1}0}} m $\\
  161 &  &  & $\overline{4}2m$ & ${}^{\overline{4}^{+}} 4 {}^{m_{1\overline{1}0}} m {}^{2_{010}} m $\\
  162 & $\overline{4}2m$ & $\overline{4}2m$ & $1$ & ${}^{1} \overline{4} {}^{1} 2 {}^{1} m $\\
  163 &  & $\overline{4}$ & $2$ & ${}^{1} \overline{4} {}^{2} 2 {}^{2} m $\\
  164 &  &  & $m$ & ${}^{1} \overline{4} {}^{m} 2 {}^{m} m $\\
  165 &  &  & $\overline{1}$ & ${}^{1} \overline{4} {}^{\overline{1}} 2 {}^{\overline{1}} m $\\
  166 &  & $mm2$ & $2$ & ${}^{2} \overline{4} {}^{2} 2 {}^{1} m $\\
  167 &  &  & $m$ & ${}^{m} \overline{4} {}^{m} 2 {}^{1} m $\\
  168 &  &  & $\overline{1}$ & ${}^{\overline{1}} \overline{4} {}^{\overline{1}} 2 {}^{1} m $\\
  169 &  & $222$ & $2$ & ${}^{2} \overline{4} {}^{1} 2 {}^{2} m $\\
  170 &  &  & $m$ & ${}^{m} \overline{4} {}^{1} 2 {}^{m} m $\\
  171 &  &  & $\overline{1}$ & ${}^{\overline{1}} \overline{4} {}^{1} 2 {}^{\overline{1}} m $\\
  172 &  & $2$ & $222$ & ${}^{2_{100}} \overline{4} {}^{2_{010}} 2 {}^{2_{001}} m $\\
  173 &  &  & $mm2$ & ${}^{2} \overline{4} {}^{m_{100}} 2 {}^{m_{010}} m $\\
  174 &  &  &  & ${}^{m_{100}} \overline{4} {}^{2} 2 {}^{m_{010}} m $\\
  175 &  &  &  & ${}^{m_{100}} \overline{4} {}^{m_{010}} 2 {}^{2} m $\\
  176 &  &  & $2/m$ & ${}^{2} \overline{4} {}^{\overline{1}} 2 {}^{m} m $\\
  177 &  &  &  & ${}^{2} \overline{4} {}^{m} 2 {}^{\overline{1}} m $\\
  178 &  &  &  & ${}^{\overline{1}} \overline{4} {}^{2} 2 {}^{m} m $\\
  179 &  &  &  & ${}^{\overline{1}} \overline{4} {}^{m} 2 {}^{2} m $\\
  180 &  &  &  & ${}^{m} \overline{4} {}^{2} 2 {}^{\overline{1}} m $\\
  181 &  &  &  & ${}^{m} \overline{4} {}^{\overline{1}} 2 {}^{2} m $\\
  182 &  & $1$ & $422$ & ${}^{4^{+}} \overline{4} {}^{2_{100}} 2 {}^{2_{1\overline{1}0}} m $\\
  183 &  &  & $4mm$ & ${}^{4^{+}} \overline{4} {}^{m_{100}} 2 {}^{m_{1\overline{1}0}} m $\\
  184 &  &  & $\overline{4}2m$ & ${}^{\overline{4}^{+}} \overline{4} {}^{2_{100}} 2 {}^{m_{1\overline{1}0}} m $\\
  185 &  &  &  & ${}^{\overline{4}^{+}} \overline{4} {}^{m_{1\overline{1}0}} 2 {}^{2_{010}} m $\\
  186 & $4/mmm$ & $4/mmm$ & $1$ & ${}^{1} 4  / {}^{1} m {}^{1} m {}^{1} m $\\
  187 &  & $\overline{4}2m$ & $2$ & ${}^{2} 4  / {}^{2} m {}^{2} m {}^{1} m $\\
  188 &  &  & $m$ & ${}^{m} 4  / {}^{m} m {}^{m} m {}^{1} m $\\
  189 &  &  & $\overline{1}$ & ${}^{\overline{1}} 4  / {}^{\overline{1}} m {}^{\overline{1}} m {}^{1} m $\\
  190 &  & $4mm$ & $2$ & ${}^{1} 4  / {}^{2} m {}^{1} m {}^{1} m $\\
  191 &  &  & $m$ & ${}^{1} 4  / {}^{m} m {}^{1} m {}^{1} m $\\
  192 &  &  & $\overline{1}$ & ${}^{1} 4  / {}^{\overline{1}} m {}^{1} m {}^{1} m $\\
  193 &  & $mmm$ & $2$ & ${}^{2} 4  / {}^{1} m {}^{1} m {}^{2} m $\\
  194 &  &  & $m$ & ${}^{m} 4  / {}^{1} m {}^{1} m {}^{m} m $\\
  195 &  &  & $\overline{1}$ & ${}^{\overline{1}} 4  / {}^{1} m {}^{1} m {}^{\overline{1}} m $\\
  196 &  & $4/m$ & $2$ & ${}^{1} 4  / {}^{1} m {}^{2} m {}^{2} m $\\
  197 &  &  & $m$ & ${}^{1} 4  / {}^{1} m {}^{m} m {}^{m} m $\\
  198 &  &  & $\overline{1}$ & ${}^{1} 4  / {}^{1} m {}^{\overline{1}} m {}^{\overline{1}} m $\\
  199 &  & $422$ & $2$ & ${}^{1} 4  / {}^{2} m {}^{2} m {}^{2} m $\\
  200 &  &  & $m$ & ${}^{1} 4  / {}^{m} m {}^{m} m {}^{m} m $\\
  201 &  &  & $\overline{1}$ & ${}^{1} 4  / {}^{\overline{1}} m {}^{\overline{1}} m {}^{\overline{1}} m $\\
  202 &  & $\overline{4}$ & $222$ & ${}^{2_{001}} 4  / {}^{2_{001}} m {}^{2_{100}} m {}^{2_{010}} m $\\
  203 &  &  & $mm2$ & ${}^{2} 4  / {}^{2} m {}^{m_{100}} m {}^{m_{010}} m $\\
  204 &  &  &  & ${}^{m_{100}} 4  / {}^{m_{100}} m {}^{2} m {}^{m_{010}} m $\\
  205 &  &  & $2/m$ & ${}^{2} 4  / {}^{2} m {}^{\overline{1}} m {}^{m} m $\\
  206 &  &  &  & ${}^{\overline{1}} 4  / {}^{\overline{1}} m {}^{2} m {}^{m} m $\\
  207 &  &  &  & ${}^{m} 4  / {}^{m} m {}^{2} m {}^{\overline{1}} m $\\
  208 &  & $4$ & $222$ & ${}^{1} 4  / {}^{2_{100}} m {}^{2_{010}} m {}^{2_{010}} m $\\
  209 &  &  & $mm2$ & ${}^{1} 4  / {}^{2} m {}^{m_{100}} m {}^{m_{100}} m $\\
  210 &  &  &  & ${}^{1} 4  / {}^{m_{100}} m {}^{2} m {}^{2} m $\\
  211 &  &  &  & ${}^{1} 4  / {}^{m_{100}} m {}^{m_{010}} m {}^{m_{010}} m $\\
  212 &  &  & $2/m$ & ${}^{1} 4  / {}^{2} m {}^{\overline{1}} m {}^{\overline{1}} m $\\
  213 &  &  &  & ${}^{1} 4  / {}^{2} m {}^{m} m {}^{m} m $\\
  214 &  &  &  & ${}^{1} 4  / {}^{\overline{1}} m {}^{2} m {}^{2} m $\\
  215 &  &  &  & ${}^{1} 4  / {}^{\overline{1}} m {}^{m} m {}^{m} m $\\
  216 &  &  &  & ${}^{1} 4  / {}^{m} m {}^{2} m {}^{2} m $\\
  217 &  &  &  & ${}^{1} 4  / {}^{m} m {}^{\overline{1}} m {}^{\overline{1}} m $\\
  218 &  & $2/m$ & $222$ & ${}^{2_{001}} 4  / {}^{1} m {}^{2_{100}} m {}^{2_{010}} m $\\
  219 &  &  & $mm2$ & ${}^{2} 4  / {}^{1} m {}^{m_{100}} m {}^{m_{010}} m $\\
  220 &  &  &  & ${}^{m_{100}} 4  / {}^{1} m {}^{2} m {}^{m_{010}} m $\\
  221 &  &  & $2/m$ & ${}^{2} 4  / {}^{1} m {}^{\overline{1}} m {}^{m} m $\\
  222 &  &  &  & ${}^{\overline{1}} 4  / {}^{1} m {}^{2} m {}^{m} m $\\
  223 &  &  &  & ${}^{m} 4  / {}^{1} m {}^{2} m {}^{\overline{1}} m $\\
  224 &  & $mm2$ & $222$ & ${}^{2_{001}} 4  / {}^{2_{100}} m {}^{1} m {}^{2_{001}} m $\\
  225 &  &  & $mm2$ & ${}^{2} 4  / {}^{m_{100}} m {}^{1} m {}^{2} m $\\
  226 &  &  &  & ${}^{m_{100}} 4  / {}^{2} m {}^{1} m {}^{m_{100}} m $\\
  227 &  &  &  & ${}^{m_{100}} 4  / {}^{m_{010}} m {}^{1} m {}^{m_{100}} m $\\
  228 &  &  & $2/m$ & ${}^{2} 4  / {}^{\overline{1}} m {}^{1} m {}^{2} m $\\
  229 &  &  &  & ${}^{2} 4  / {}^{m} m {}^{1} m {}^{2} m $\\
  230 &  &  &  & ${}^{\overline{1}} 4  / {}^{2} m {}^{1} m {}^{\overline{1}} m $\\
  231 &  &  &  & ${}^{\overline{1}} 4  / {}^{m} m {}^{1} m {}^{\overline{1}} m $\\
  232 &  &  &  & ${}^{m} 4  / {}^{2} m {}^{1} m {}^{m} m $\\
  233 &  &  &  & ${}^{m} 4  / {}^{\overline{1}} m {}^{1} m {}^{m} m $\\
  234 &  & $222$ & $222$ & ${}^{2_{001}} 4  / {}^{2_{100}} m {}^{2_{100}} m {}^{2_{010}} m $\\
  235 &  &  & $mm2$ & ${}^{2} 4  / {}^{m_{100}} m {}^{m_{100}} m {}^{m_{010}} m $\\
  236 &  &  &  & ${}^{m_{100}} 4  / {}^{2} m {}^{2} m {}^{m_{010}} m $\\
  237 &  &  &  & ${}^{m_{100}} 4  / {}^{m_{010}} m {}^{m_{010}} m {}^{2} m $\\
  238 &  &  & $2/m$ & ${}^{2} 4  / {}^{\overline{1}} m {}^{\overline{1}} m {}^{m} m $\\
  239 &  &  &  & ${}^{2} 4  / {}^{m} m {}^{m} m {}^{\overline{1}} m $\\
  240 &  &  &  & ${}^{\overline{1}} 4  / {}^{2} m {}^{2} m {}^{m} m $\\
  241 &  &  &  & ${}^{\overline{1}} 4  / {}^{m} m {}^{m} m {}^{2} m $\\
  242 &  &  &  & ${}^{m} 4  / {}^{2} m {}^{2} m {}^{\overline{1}} m $\\
  243 &  &  &  & ${}^{m} 4  / {}^{\overline{1}} m {}^{\overline{1}} m {}^{2} m $\\
  244 &  & $m$ & $422$ & ${}^{4^{+}} 4  / {}^{1} m {}^{2_{100}} m {}^{2_{1\overline{1}0}} m $\\
  245 &  &  & $4mm$ & ${}^{4^{+}} 4  / {}^{1} m {}^{m_{100}} m {}^{m_{1\overline{1}0}} m $\\
  246 &  &  & $\overline{4}2m$ & ${}^{\overline{4}^{+}} 4  / {}^{1} m {}^{2_{100}} m {}^{m_{1\overline{1}0}} m $\\
  247 &  & $\overline{1}$ & $422$ & ${}^{4^{+}} 4  / {}^{2_{001}} m {}^{2_{100}} m {}^{2_{1\overline{1}0}} m $\\
  248 &  &  & $4mm$ & ${}^{4^{+}} 4  / {}^{2} m {}^{m_{100}} m {}^{m_{1\overline{1}0}} m $\\
  249 &  &  & $\overline{4}2m$ & ${}^{\overline{4}^{+}} 4  / {}^{2_{001}} m {}^{m_{1\overline{1}0}} m {}^{2_{010}} m $\\
  250 &  & $2$ & $mmm$ & ${}^{m_{001}} 4  / {}^{2_{100}} m {}^{m_{100}} m {}^{2_{010}} m $\\
  251 &  &  &  & ${}^{m_{001}} 4  / {}^{m_{100}} m {}^{2_{100}} m {}^{m_{010}} m $\\
  252 &  &  &  & ${}^{2_{001}} 4  / {}^{m_{100}} m {}^{2_{100}} m {}^{2_{010}} m $\\
  253 &  &  &  & ${}^{m_{001}} 4  / {}^{2_{001}} m {}^{2_{100}} m {}^{m_{010}} m $\\
  254 &  &  &  & ${}^{2_{001}} 4  / {}^{m_{001}} m {}^{2_{100}} m {}^{2_{010}} m $\\
  255 &  &  &  & ${}^{2_{001}} 4  / {}^{m_{001}} m {}^{m_{100}} m {}^{m_{010}} m $\\
  256 &  &  &  & ${}^{m_{001}} 4  / {}^{2_{100}} m {}^{2_{001}} m {}^{\overline{1}} m $\\
  257 &  &  &  & ${}^{m_{001}} 4  / {}^{m_{100}} m {}^{\overline{1}} m {}^{2_{001}} m $\\
  258 &  &  &  & ${}^{m_{001}} 4  / {}^{\overline{1}} m {}^{m_{100}} m {}^{2_{010}} m $\\
  259 &  &  &  & ${}^{2_{001}} 4  / {}^{2_{100}} m {}^{m_{100}} m {}^{m_{010}} m $\\
  260 &  &  &  & ${}^{2_{001}} 4  / {}^{2_{100}} m {}^{\overline{1}} m {}^{m_{001}} m $\\
  261 &  &  &  & ${}^{2_{001}} 4  / {}^{\overline{1}} m {}^{m_{100}} m {}^{m_{010}} m $\\
  262 &  &  &  & ${}^{2_{001}} 4  / {}^{m_{100}} m {}^{\overline{1}} m {}^{m_{001}} m $\\
  263 &  &  &  & ${}^{2_{001}} 4  / {}^{\overline{1}} m {}^{2_{100}} m {}^{2_{010}} m $\\
  264 &  &  &  & ${}^{\overline{1}} 4  / {}^{2_{100}} m {}^{2_{001}} m {}^{m_{001}} m $\\
  265 &  &  &  & ${}^{\overline{1}} 4  / {}^{m_{100}} m {}^{m_{001}} m {}^{2_{001}} m $\\
  266 &  & $1$ & $4/mmm$ & ${}^{4^{+}} 4  / {}^{m_{001}} m {}^{m_{100}} m {}^{m_{1\overline{1}0}} m $\\
  267 &  &  &  & ${}^{4^{+}} 4  / {}^{\overline{1}} m {}^{m_{100}} m {}^{m_{1\overline{1}0}} m $\\
  268 &  &  &  & ${}^{4^{+}} 4  / {}^{m_{001}} m {}^{2_{100}} m {}^{2_{1\overline{1}0}} m $\\
  269 &  &  &  & ${}^{4^{+}} 4  / {}^{\overline{1}} m {}^{2_{100}} m {}^{2_{1\overline{1}0}} m $\\
  270 &  &  &  & ${}^{\overline{4}^{+}} 4  / {}^{m_{001}} m {}^{m_{100}} m {}^{2_{1\overline{1}0}} m $\\
  271 &  &  &  & ${}^{\overline{4}^{+}} 4  / {}^{\overline{1}} m {}^{m_{100}} m {}^{2_{1\overline{1}0}} m $\\
  272 & $3$ & $3$ & $1$ & ${}^{1} 3 $\\
  273 &  & $1$ & $3$ & ${}^{3^{+}} 3 $\\
  274 & $\overline{3}$ & $\overline{3}$ & $1$ & ${}^{1} \overline{3} $\\
  275 &  & $3$ & $2$ & ${}^{2} \overline{3} $\\
  276 &  &  & $m$ & ${}^{m} \overline{3} $\\
  277 &  &  & $\overline{1}$ & ${}^{\overline{1}} \overline{3} $\\
  278 &  & $\overline{1}$ & $3$ & ${}^{3^{+}} \overline{3} $\\
  279 &  & $1$ & $6$ & ${}^{6^{+}} \overline{3} $\\
  280 &  &  & $\overline{3}$ & ${}^{\overline{3}^{+}} \overline{3} $\\
  281 &  &  & $\overline{6}$ & ${}^{\overline{6}^{+}} \overline{3} $\\
  282 & $32$ & $32$ & $1$ & ${}^{1} 3 {}^{1} 2 $\\
  283 &  & $3$ & $2$ & ${}^{1} 3 {}^{2} 2 $\\
  284 &  &  & $m$ & ${}^{1} 3 {}^{m} 2 $\\
  285 &  &  & $\overline{1}$ & ${}^{1} 3 {}^{\overline{1}} 2 $\\
  286 &  & $1$ & $32$ & ${}^{3^{+}} 3 {}^{2_{1\overline{1}0}} 2 $\\
  287 &  &  & $3m$ & ${}^{3^{+}} 3 {}^{m_{100}} 2 $\\
  288 & $3m$ & $3m$ & $1$ & ${}^{1} 3 {}^{1} m $\\
  289 &  & $3$ & $2$ & ${}^{1} 3 {}^{2} m $\\
  290 &  &  & $m$ & ${}^{1} 3 {}^{m} m $\\
  291 &  &  & $\overline{1}$ & ${}^{1} 3 {}^{\overline{1}} m $\\
  292 &  & $1$ & $32$ & ${}^{3^{+}} 3 {}^{2_{1\overline{1}0}} m $\\
  293 &  &  & $3m$ & ${}^{3^{+}} 3 {}^{m_{100}} m $\\
  294 & $\overline{3}m$ & $\overline{3}m$ & $1$ & ${}^{1} \overline{3} {}^{1} m $\\
  295 &  & $\overline{3}$ & $2$ & ${}^{1} \overline{3} {}^{2} m $\\
  296 &  &  & $m$ & ${}^{1} \overline{3} {}^{m} m $\\
  297 &  &  & $\overline{1}$ & ${}^{1} \overline{3} {}^{\overline{1}} m $\\
  298 &  & $3m$ & $2$ & ${}^{2} \overline{3} {}^{1} m $\\
  299 &  &  & $m$ & ${}^{m} \overline{3} {}^{1} m $\\
  300 &  &  & $\overline{1}$ & ${}^{\overline{1}} \overline{3} {}^{1} m $\\
  301 &  & $32$ & $2$ & ${}^{2} \overline{3} {}^{2} m $\\
  302 &  &  & $m$ & ${}^{m} \overline{3} {}^{m} m $\\
  303 &  &  & $\overline{1}$ & ${}^{\overline{1}} \overline{3} {}^{\overline{1}} m $\\
  304 &  & $3$ & $222$ & ${}^{2_{100}} \overline{3} {}^{2_{001}} m $\\
  305 &  &  & $mm2$ & ${}^{2} \overline{3} {}^{m_{100}} m $\\
  306 &  &  &  & ${}^{m_{100}} \overline{3} {}^{m_{010}} m $\\
  307 &  &  &  & ${}^{m_{100}} \overline{3} {}^{2} m $\\
  308 &  &  & $2/m$ & ${}^{2} \overline{3} {}^{m} m $\\
  309 &  &  &  & ${}^{2} \overline{3} {}^{\overline{1}} m $\\
  310 &  &  &  & ${}^{\overline{1}} \overline{3} {}^{m} m $\\
  311 &  &  &  & ${}^{\overline{1}} \overline{3} {}^{2} m $\\
  312 &  &  &  & ${}^{m} \overline{3} {}^{\overline{1}} m $\\
  313 &  &  &  & ${}^{m} \overline{3} {}^{2} m $\\
  314 &  & $\overline{1}$ & $32$ & ${}^{3^{+}} \overline{3} {}^{2_{1\overline{1}0}} m $\\
  315 &  &  & $3m$ & ${}^{3^{+}} \overline{3} {}^{m_{100}} m $\\
  316 &  & $1$ & $622$ & ${}^{6^{+}} \overline{3} {}^{2_{100}} m $\\
  317 &  &  & $\overline{3}m$ & ${}^{\overline{3}^{+}} \overline{3} {}^{m_{1\overline{1}0}} m $\\
  318 &  &  &  & ${}^{\overline{3}^{+}} \overline{3} {}^{2_{1\overline{1}0}} m $\\
  319 &  &  & $6mm$ & ${}^{6^{+}} \overline{3} {}^{m_{100}} m $\\
  320 &  &  & $\overline{6}m2$ & ${}^{6^{+}} \overline{3} {}^{m_{100}} m $\\
  321 &  &  &  & ${}^{6^{+}} \overline{3} {}^{2_{1\overline{1}0}} m $\\
  322 & $\overline{6}$ & $\overline{6}$ & $1$ & ${}^{1} \overline{6} $\\
  323 &  & $3$ & $2$ & ${}^{2} \overline{6} $\\
  324 &  &  & $m$ & ${}^{m} \overline{6} $\\
  325 &  &  & $\overline{1}$ & ${}^{\overline{1}} \overline{6} $\\
  326 &  & $m$ & $3$ & ${}^{3^{+}} \overline{6} $\\
  327 &  & $1$ & $6$ & ${}^{6^{+}} \overline{6} $\\
  328 &  &  & $\overline{3}$ & ${}^{\overline{3}^{+}} \overline{6} $\\
  329 &  &  & $\overline{6}$ & ${}^{\overline{6}^{+}} \overline{6} $\\
  330 & $6$ & $6$ & $1$ & ${}^{1} 6 $\\
  331 &  & $3$ & $2$ & ${}^{2} 6 $\\
  332 &  &  & $m$ & ${}^{m} 6 $\\
  333 &  &  & $\overline{1}$ & ${}^{\overline{1}} 6 $\\
  334 &  & $2$ & $3$ & ${}^{3^{+}} 6 $\\
  335 &  & $1$ & $6$ & ${}^{6^{+}} 6 $\\
  336 &  &  & $\overline{3}$ & ${}^{\overline{3}^{+}} 6 $\\
  337 &  &  & $\overline{6}$ & ${}^{\overline{6}^{+}} 6 $\\
  338 & $622$ & $622$ & $1$ & ${}^{1} 6 {}^{1} 2 {}^{1} 2 $\\
  339 &  & $6$ & $2$ & ${}^{1} 6 {}^{2} 2 {}^{2} 2 $\\
  340 &  &  & $m$ & ${}^{1} 6 {}^{m} 2 {}^{m} 2 $\\
  341 &  &  & $\overline{1}$ & ${}^{1} 6 {}^{\overline{1}} 2 {}^{\overline{1}} 2 $\\
  342 &  & $32$ & $2$ & ${}^{2} 6 {}^{1} 2 {}^{2} 2 $\\
  343 &  &  & $m$ & ${}^{m} 6 {}^{1} 2 {}^{m} 2 $\\
  344 &  &  & $\overline{1}$ & ${}^{\overline{1}} 6 {}^{1} 2 {}^{\overline{1}} 2 $\\
  345 &  & $3$ & $222$ & ${}^{2_{100}} 6 {}^{2_{010}} 2 {}^{2_{001}} 2 $\\
  346 &  &  & $mm2$ & ${}^{2} 6 {}^{m_{100}} 2 {}^{m_{010}} 2 $\\
  347 &  &  &  & ${}^{m_{100}} 6 {}^{2} 2 {}^{m_{010}} 2 $\\
  348 &  &  & $2/m$ & ${}^{2} 6 {}^{\overline{1}} 2 {}^{m} 2 $\\
  349 &  &  &  & ${}^{\overline{1}} 6 {}^{2} 2 {}^{m} 2 $\\
  350 &  &  &  & ${}^{m} 6 {}^{2} 2 {}^{\overline{1}} 2 $\\
  351 &  & $2$ & $32$ & ${}^{3^{+}} 6 {}^{2_{1\overline{1}0}} 2 {}^{2_{120}} 2 $\\
  352 &  &  & $3m$ & ${}^{3^{+}} 6 {}^{m_{100}} 2 {}^{m_{010}} 2 $\\
  353 &  & $1$ & $622$ & ${}^{6^{+}} 6 {}^{2_{100}} 2 {}^{2_{1\overline{1}0}} 2 $\\
  354 &  &  & $\overline{3}m$ & ${}^{\overline{3}^{+}} 6 {}^{2_{1\overline{1}0}} 2 {}^{m_{120}} 2 $\\
  355 &  &  & $6mm$ & ${}^{6^{+}} 6 {}^{m_{100}} 2 {}^{m_{1\overline{1}0}} 2 $\\
  356 &  &  & $\overline{6}m2$ & ${}^{6^{+}} 6 {}^{m_{100}} 2 {}^{2_{1\overline{1}0}} 2 $\\
  357 & $6/m$ & $6/m$ & $1$ & ${}^{1} 6  / {}^{1} m $\\
  358 &  & $\overline{3}$ & $2$ & ${}^{2} 6  / {}^{2} m $\\
  359 &  &  & $m$ & ${}^{m} 6  / {}^{m} m $\\
  360 &  &  & $\overline{1}$ & ${}^{\overline{1}} 6  / {}^{\overline{1}} m $\\
  361 &  & $\overline{6}$ & $2$ & ${}^{2} 6  / {}^{1} m $\\
  362 &  &  & $m$ & ${}^{m} 6  / {}^{1} m $\\
  363 &  &  & $\overline{1}$ & ${}^{\overline{1}} 6  / {}^{1} m $\\
  364 &  & $6$ & $2$ & ${}^{1} 6  / {}^{2} m $\\
  365 &  &  & $m$ & ${}^{1} 6  / {}^{m} m $\\
  366 &  &  & $\overline{1}$ & ${}^{1} 6  / {}^{\overline{1}} m $\\
  367 &  & $2/m$ & $3$ & ${}^{3^{+}} 6  / {}^{1} m $\\
  368 &  & $3$ & $222$ & ${}^{2_{001}} 6  / {}^{2_{100}} m $\\
  369 &  &  & $mm2$ & ${}^{2} 6  / {}^{m_{100}} m $\\
  370 &  &  &  & ${}^{m_{100}} 6  / {}^{2} m $\\
  371 &  &  &  & ${}^{m_{100}} 6  / {}^{m_{010}} m $\\
  372 &  &  & $2/m$ & ${}^{2} 6  / {}^{\overline{1}} m $\\
  373 &  &  &  & ${}^{2} 6  / {}^{m} m $\\
  374 &  &  &  & ${}^{\overline{1}} 6  / {}^{2} m $\\
  375 &  &  &  & ${}^{\overline{1}} 6  / {}^{m} m $\\
  376 &  &  &  & ${}^{m} 6  / {}^{2} m $\\
  377 &  &  &  & ${}^{m} 6  / {}^{\overline{1}} m $\\
  378 &  & $2$ & $6$ & ${}^{3^{+}} 6  / {}^{2} m $\\
  379 &  &  & $\overline{3}$ & ${}^{\overline{3}^{+}} 6  / {}^{\overline{1}} m $\\
  380 &  &  & $\overline{6}$ & ${}^{3^{+}} 6  / {}^{m} m $\\
  381 &  & $m$ & $6$ & ${}^{6^{+}} 6  / {}^{1} m $\\
  382 &  &  & $\overline{3}$ & ${}^{\overline{3}^{+}} 6  / {}^{1} m $\\
  383 &  &  & $\overline{6}$ & ${}^{\overline{6}^{+}} 6  / {}^{1} m $\\
  384 &  & $\overline{1}$ & $6$ & ${}^{6^{+}} 6  / {}^{2} m $\\
  385 &  &  & $\overline{3}$ & ${}^{\overline{3}^{+}} 6  / {}^{\overline{1}} m $\\
  386 &  &  & $\overline{6}$ & ${}^{\overline{6}^{+}} 6  / {}^{m} m $\\
  387 &  & $1$ & $6/m$ & ${}^{6^{+}} 6  / {}^{m} m $\\
  388 &  &  &  & ${}^{6^{+}} 6  / {}^{\overline{1}} m $\\
  389 &  &  &  & ${}^{\overline{3}^{+}} 6  / {}^{m} m $\\
  390 &  &  &  & ${}^{\overline{3}^{+}} 6  / {}^{2} m $\\
  391 &  &  &  & ${}^{\overline{6}^{+}} 6  / {}^{\overline{1}} m $\\
  392 &  &  &  & ${}^{\overline{6}^{+}} 6  / {}^{2} m $\\
  393 & $6mm$ & $6mm$ & $1$ & ${}^{1} 6 {}^{1} m {}^{1} m $\\
  394 &  & $6$ & $2$ & ${}^{1} 6 {}^{2} m {}^{2} m $\\
  395 &  &  & $m$ & ${}^{1} 6 {}^{m} m {}^{m} m $\\
  396 &  &  & $\overline{1}$ & ${}^{1} 6 {}^{\overline{1}} m {}^{\overline{1}} m $\\
  397 &  & $3m$ & $2$ & ${}^{2} 6 {}^{1} m {}^{2} m $\\
  398 &  &  & $m$ & ${}^{m} 6 {}^{1} m {}^{m} m $\\
  399 &  &  & $\overline{1}$ & ${}^{\overline{1}} 6 {}^{1} m {}^{\overline{1}} m $\\
  400 &  & $3$ & $222$ & ${}^{2_{100}} 6 {}^{2_{010}} m {}^{2_{001}} m $\\
  401 &  &  & $mm2$ & ${}^{2} 6 {}^{m_{100}} m {}^{m_{010}} m $\\
  402 &  &  &  & ${}^{m_{100}} 6 {}^{2} m {}^{m_{010}} m $\\
  403 &  &  & $2/m$ & ${}^{2} 6 {}^{\overline{1}} m {}^{m} m $\\
  404 &  &  &  & ${}^{\overline{1}} 6 {}^{2} m {}^{m} m $\\
  405 &  &  &  & ${}^{m} 6 {}^{2} m {}^{\overline{1}} m $\\
  406 &  & $2$ & $32$ & ${}^{3^{+}} 6 {}^{2_{1\overline{1}0}} m {}^{2_{120}} m $\\
  407 &  &  & $3m$ & ${}^{3^{+}} 6 {}^{m_{100}} m {}^{m_{010}} m $\\
  408 &  & $1$ & $622$ & ${}^{6^{+}} 6 {}^{2_{100}} m {}^{2_{1\overline{1}0}} m $\\
  409 &  &  & $\overline{3}m$ & ${}^{\overline{3}^{+}} 6 {}^{2_{1\overline{1}0}} m {}^{m_{120}} m $\\
  410 &  &  & $6mm$ & ${}^{6^{+}} 6 {}^{m_{100}} m {}^{m_{1\overline{1}0}} m $\\
  411 &  &  & $\overline{6}m2$ & ${}^{6^{+}} 6 {}^{m_{100}} m {}^{2_{1\overline{1}0}} m $\\
  412 & $\overline{6}m2$ & $\overline{6}m2$ & $1$ & ${}^{1} \overline{6} {}^{1} m {}^{1} 2 $\\
  413 &  & $\overline{6}$ & $2$ & ${}^{1} \overline{6} {}^{2} m {}^{2} 2 $\\
  414 &  &  & $m$ & ${}^{1} \overline{6} {}^{m} m {}^{m} 2 $\\
  415 &  &  & $\overline{1}$ & ${}^{1} \overline{6} {}^{\overline{1}} m {}^{\overline{1}} 2 $\\
  416 &  & $3m$ & $2$ & ${}^{2} \overline{6} {}^{1} m {}^{2} 2 $\\
  417 &  &  & $m$ & ${}^{m} \overline{6} {}^{1} m {}^{m} 2 $\\
  418 &  &  & $\overline{1}$ & ${}^{\overline{1}} \overline{6} {}^{1} m {}^{\overline{1}} 2 $\\
  419 &  & $32$ & $2$ & ${}^{2} \overline{6} {}^{2} m {}^{1} 2 $\\
  420 &  &  & $m$ & ${}^{m} \overline{6} {}^{m} m {}^{1} 2 $\\
  421 &  &  & $\overline{1}$ & ${}^{\overline{1}} \overline{6} {}^{\overline{1}} m {}^{1} 2 $\\
  422 &  & $3$ & $222$ & ${}^{2_{001}} \overline{6} {}^{2_{100}} m {}^{2_{010}} 2 $\\
  423 &  &  & $mm2$ & ${}^{2} \overline{6} {}^{m_{100}} m {}^{m_{010}} 2 $\\
  424 &  &  &  & ${}^{m_{100}} \overline{6} {}^{m_{010}} m {}^{2} 2 $\\
  425 &  &  &  & ${}^{m_{100}} \overline{6} {}^{2} m {}^{m_{010}} 2 $\\
  426 &  &  & $2/m$ & ${}^{2} \overline{6} {}^{m} m {}^{\overline{1}} 2 $\\
  427 &  &  &  & ${}^{2} \overline{6} {}^{\overline{1}} m {}^{m} 2 $\\
  428 &  &  &  & ${}^{\overline{1}} \overline{6} {}^{m} m {}^{2} 2 $\\
  429 &  &  &  & ${}^{\overline{1}} \overline{6} {}^{2} m {}^{m} 2 $\\
  430 &  &  &  & ${}^{m} \overline{6} {}^{\overline{1}} m {}^{2} 2 $\\
  431 &  &  &  & ${}^{m} \overline{6} {}^{2} m {}^{\overline{1}} 2 $\\
  432 &  & $m$ & $32$ & ${}^{3^{+}} \overline{6} {}^{2_{1\overline{1}0}} m {}^{2_{120}} 2 $\\
  433 &  &  & $3m$ & ${}^{3^{+}} \overline{6} {}^{m_{100}} m {}^{m_{010}} 2 $\\
  434 &  & $1$ & $622$ & ${}^{6^{+}} \overline{6} {}^{2_{100}} m {}^{2_{1\overline{1}0}} 2 $\\
  435 &  &  & $\overline{3}m$ & ${}^{\overline{3}^{+}} \overline{6} {}^{m_{1\overline{1}0}} m {}^{2_{120}} 2 $\\
  436 &  &  &  & ${}^{\overline{3}^{+}} \overline{6} {}^{2_{1\overline{1}0}} m {}^{m_{120}} 2 $\\
  437 &  &  & $6mm$ & ${}^{6^{+}} \overline{6} {}^{m_{100}} m {}^{m_{1\overline{1}0}} 2 $\\
  438 &  &  & $\overline{6}m2$ & ${}^{6^{+}} \overline{6} {}^{m_{100}} m {}^{2_{1\overline{1}0}} 2 $\\
  439 &  &  &  & ${}^{6^{+}} \overline{6} {}^{2_{1\overline{1}0}} m {}^{m_{010}} 2 $\\
  440 & $6/mmm$ & $6/mmm$ & $1$ & ${}^{1} 6  / {}^{1} m {}^{1} m {}^{1} m $\\
  441 &  & $\overline{3}m$ & $2$ & ${}^{2} 6  / {}^{2} m {}^{1} m {}^{2} m $\\
  442 &  &  & $m$ & ${}^{m} 6  / {}^{m} m {}^{1} m {}^{m} m $\\
  443 &  &  & $\overline{1}$ & ${}^{\overline{1}} 6  / {}^{\overline{1}} m {}^{1} m {}^{\overline{1}} m $\\
  444 &  & $\overline{6}m2$ & $2$ & ${}^{2} 6  / {}^{1} m {}^{2} m {}^{1} m $\\
  445 &  &  & $m$ & ${}^{m} 6  / {}^{1} m {}^{m} m {}^{1} m $\\
  446 &  &  & $\overline{1}$ & ${}^{\overline{1}} 6  / {}^{1} m {}^{\overline{1}} m {}^{1} m $\\
  447 &  & $6/m$ & $2$ & ${}^{1} 6  / {}^{1} m {}^{2} m {}^{2} m $\\
  448 &  &  & $m$ & ${}^{1} 6  / {}^{1} m {}^{m} m {}^{m} m $\\
  449 &  &  & $\overline{1}$ & ${}^{1} 6  / {}^{1} m {}^{\overline{1}} m {}^{\overline{1}} m $\\
  450 &  & $6mm$ & $2$ & ${}^{1} 6  / {}^{2} m {}^{1} m {}^{1} m $\\
  451 &  &  & $m$ & ${}^{1} 6  / {}^{m} m {}^{1} m {}^{1} m $\\
  452 &  &  & $\overline{1}$ & ${}^{1} 6  / {}^{\overline{1}} m {}^{1} m {}^{1} m $\\
  453 &  & $622$ & $2$ & ${}^{1} 6  / {}^{2} m {}^{2} m {}^{2} m $\\
  454 &  &  & $m$ & ${}^{1} 6  / {}^{m} m {}^{m} m {}^{m} m $\\
  455 &  &  & $\overline{1}$ & ${}^{1} 6  / {}^{\overline{1}} m {}^{\overline{1}} m {}^{\overline{1}} m $\\
  456 &  & $\overline{3}$ & $222$ & ${}^{2_{001}} 6  / {}^{2_{001}} m {}^{2_{100}} m {}^{2_{010}} m $\\
  457 &  &  & $mm2$ & ${}^{2} 6  / {}^{2} m {}^{m_{100}} m {}^{m_{010}} m $\\
  458 &  &  &  & ${}^{m_{100}} 6  / {}^{m_{100}} m {}^{2} m {}^{m_{010}} m $\\
  459 &  &  & $2/m$ & ${}^{2} 6  / {}^{2} m {}^{\overline{1}} m {}^{m} m $\\
  460 &  &  &  & ${}^{\overline{1}} 6  / {}^{\overline{1}} m {}^{2} m {}^{m} m $\\
  461 &  &  &  & ${}^{m} 6  / {}^{m} m {}^{2} m {}^{\overline{1}} m $\\
  462 &  & $\overline{6}$ & $222$ & ${}^{2_{001}} 6  / {}^{1} m {}^{2_{100}} m {}^{2_{010}} m $\\
  463 &  &  & $mm2$ & ${}^{2} 6  / {}^{1} m {}^{m_{100}} m {}^{m_{010}} m $\\
  464 &  &  &  & ${}^{m_{100}} 6  / {}^{1} m {}^{2} m {}^{m_{010}} m $\\
  465 &  &  & $2/m$ & ${}^{2} 6  / {}^{1} m {}^{\overline{1}} m {}^{m} m $\\
  466 &  &  &  & ${}^{\overline{1}} 6  / {}^{1} m {}^{2} m {}^{m} m $\\
  467 &  &  &  & ${}^{m} 6  / {}^{1} m {}^{2} m {}^{\overline{1}} m $\\
  468 &  & $6$ & $222$ & ${}^{1} 6  / {}^{2_{001}} m {}^{2_{100}} m {}^{2_{100}} m $\\
  469 &  &  & $mm2$ & ${}^{1} 6  / {}^{2} m {}^{m_{100}} m {}^{m_{100}} m $\\
  470 &  &  &  & ${}^{1} 6  / {}^{m_{100}} m {}^{m_{010}} m {}^{m_{010}} m $\\
  471 &  &  &  & ${}^{1} 6  / {}^{m_{100}} m {}^{2} m {}^{2} m $\\
  472 &  &  & $2/m$ & ${}^{1} 6  / {}^{2} m {}^{m} m {}^{m} m $\\
  473 &  &  &  & ${}^{1} 6  / {}^{2} m {}^{\overline{1}} m {}^{\overline{1}} m $\\
  474 &  &  &  & ${}^{1} 6  / {}^{\overline{1}} m {}^{m} m {}^{m} m $\\
  475 &  &  &  & ${}^{1} 6  / {}^{\overline{1}} m {}^{2} m {}^{2} m $\\
  476 &  &  &  & ${}^{1} 6  / {}^{m} m {}^{\overline{1}} m {}^{\overline{1}} m $\\
  477 &  &  &  & ${}^{1} 6  / {}^{m} m {}^{2} m {}^{2} m $\\
  478 &  & $3m$ & $222$ & ${}^{2_{001}} 6  / {}^{2_{100}} m {}^{1} m {}^{2_{001}} m $\\
  479 &  &  & $mm2$ & ${}^{m_{100}} 6  / {}^{m_{010}} m {}^{1} m {}^{m_{100}} m $\\
  480 &  &  &  & ${}^{m_{100}} 6  / {}^{2} m {}^{1} m {}^{m_{100}} m $\\
  481 &  &  &  & ${}^{2} 6  / {}^{m_{100}} m {}^{1} m {}^{2} m $\\
  482 &  &  & $2/m$ & ${}^{m} 6  / {}^{\overline{1}} m {}^{1} m {}^{m} m $\\
  483 &  &  &  & ${}^{\overline{1}} 6  / {}^{m} m {}^{1} m {}^{\overline{1}} m $\\
  484 &  &  &  & ${}^{m} 6  / {}^{2} m {}^{1} m {}^{m} m $\\
  485 &  &  &  & ${}^{2} 6  / {}^{m} m {}^{1} m {}^{2} m $\\
  486 &  &  &  & ${}^{\overline{1}} 6  / {}^{2} m {}^{1} m {}^{\overline{1}} m $\\
  487 &  &  &  & ${}^{2} 6  / {}^{\overline{1}} m {}^{1} m {}^{2} m $\\
  488 &  & $32$ & $222$ & ${}^{2_{001}} 6  / {}^{2_{100}} m {}^{2_{010}} m {}^{2_{100}} m $\\
  489 &  &  & $mm2$ & ${}^{m_{100}} 6  / {}^{2} m {}^{m_{010}} m {}^{2} m $\\
  490 &  &  &  & ${}^{m_{100}} 6  / {}^{m_{010}} m {}^{2} m {}^{m_{010}} m $\\
  491 &  &  &  & ${}^{2} 6  / {}^{m_{100}} m {}^{m_{010}} m {}^{m_{100}} m $\\
  492 &  &  & $2/m$ & ${}^{m} 6  / {}^{2} m {}^{\overline{1}} m {}^{2} m $\\
  493 &  &  &  & ${}^{\overline{1}} 6  / {}^{2} m {}^{m} m {}^{2} m $\\
  494 &  &  &  & ${}^{m} 6  / {}^{\overline{1}} m {}^{2} m {}^{\overline{1}} m $\\
  495 &  &  &  & ${}^{2} 6  / {}^{\overline{1}} m {}^{m} m {}^{\overline{1}} m $\\
  496 &  &  &  & ${}^{\overline{1}} 6  / {}^{m} m {}^{2} m {}^{m} m $\\
  497 &  &  &  & ${}^{2} 6  / {}^{m} m {}^{\overline{1}} m {}^{m} m $\\
  498 &  & $2/m$ & $32$ & ${}^{3^{+}} 6  / {}^{1} m {}^{2_{1\overline{1}0}} m {}^{2_{120}} m $\\
  499 &  &  & $3m$ & ${}^{3^{+}} 6  / {}^{1} m {}^{m_{100}} m {}^{m_{010}} m $\\
  500 &  & $3$ & $mmm$ & ${}^{m_{001}} 6  / {}^{2_{001}} m {}^{m_{100}} m {}^{2_{010}} m $\\
  501 &  &  &  & ${}^{2_{001}} 6  / {}^{m_{001}} m {}^{m_{100}} m {}^{m_{010}} m $\\
  502 &  &  &  & ${}^{2_{001}} 6  / {}^{m_{001}} m {}^{2_{100}} m {}^{2_{010}} m $\\
  503 &  &  &  & ${}^{m_{001}} 6  / {}^{m_{100}} m {}^{2_{100}} m {}^{m_{010}} m $\\
  504 &  &  &  & ${}^{m_{001}} 6  / {}^{2_{100}} m {}^{2_{010}} m {}^{m_{100}} m $\\
  505 &  &  &  & ${}^{2_{001}} 6  / {}^{2_{100}} m {}^{m_{010}} m {}^{m_{100}} m $\\
  506 &  &  &  & ${}^{2_{001}} 6  / {}^{m_{100}} m {}^{2_{010}} m {}^{2_{100}} m $\\
  507 &  &  &  & ${}^{m_{001}} 6  / {}^{2_{100}} m {}^{\overline{1}} m {}^{2_{001}} m $\\
  508 &  &  &  & ${}^{m_{001}} 6  / {}^{m_{100}} m {}^{2_{001}} m {}^{\overline{1}} m $\\
  509 &  &  &  & ${}^{m_{001}} 6  / {}^{\overline{1}} m {}^{2_{100}} m {}^{m_{010}} m $\\
  510 &  &  &  & ${}^{2_{001}} 6  / {}^{m_{100}} m {}^{m_{001}} m {}^{\overline{1}} m $\\
  511 &  &  &  & ${}^{2_{001}} 6  / {}^{2_{100}} m {}^{m_{001}} m {}^{\overline{1}} m $\\
  512 &  &  &  & ${}^{2_{001}} 6  / {}^{\overline{1}} m {}^{m_{100}} m {}^{m_{010}} m $\\
  513 &  &  &  & ${}^{2_{001}} 6  / {}^{\overline{1}} m {}^{2_{100}} m {}^{2_{010}} m $\\
  514 &  &  &  & ${}^{\overline{1}} 6  / {}^{m_{100}} m {}^{2_{001}} m {}^{m_{001}} m $\\
  515 &  &  &  & ${}^{\overline{1}} 6  / {}^{2_{100}} m {}^{m_{001}} m {}^{2_{001}} m $\\
  516 &  & $\overline{1}$ & $622$ & ${}^{6^{+}} 6  / {}^{2_{001}} m {}^{2_{100}} m {}^{2_{1\overline{1}0}} m $\\
  517 &  &  & $\overline{3}m$ & ${}^{\overline{3}^{+}} 6  / {}^{\overline{1}} m {}^{2_{1\overline{1}0}} m {}^{m_{120}} m $\\
  518 &  &  & $6mm$ & ${}^{6^{+}} 6  / {}^{2_{001}} m {}^{m_{100}} m {}^{m_{1\overline{1}0}} m $\\
  519 &  &  & $\overline{6}m2$ & ${}^{6^{+}} 6  / {}^{m_{001}} m {}^{m_{100}} m {}^{2_{1\overline{1}0}} m $\\
  520 &  & $m$ & $622$ & ${}^{6^{+}} 6  / {}^{1} m {}^{2_{100}} m {}^{2_{1\overline{1}0}} m $\\
  521 &  &  & $\overline{3}m$ & ${}^{\overline{3}^{+}} 6  / {}^{1} m {}^{m_{1\overline{1}0}} m {}^{2_{120}} m $\\
  522 &  &  & $6mm$ & ${}^{6^{+}} 6  / {}^{1} m {}^{m_{100}} m {}^{m_{1\overline{1}0}} m $\\
  523 &  &  & $\overline{6}m2$ & ${}^{6^{+}} 6  / {}^{1} m {}^{2_{1\overline{1}0}} m {}^{m_{010}} m $\\
  524 &  & $2$ & $622$ & ${}^{3^{+}} 6  / {}^{2_{001}} m {}^{2_{100}} m {}^{2_{010}} m $\\
  525 &  &  & $\overline{3}m$ & ${}^{3^{+}} 6  / {}^{\overline{1}} m {}^{m_{1\overline{1}0}} m {}^{m_{120}} m $\\
  526 &  &  &  & ${}^{3^{+}} 6  / {}^{\overline{1}} m {}^{2_{1\overline{1}0}} m {}^{2_{120}} m $\\
  527 &  &  & $6mm$ & ${}^{3^{+}} 6  / {}^{2_{001}} m {}^{m_{100}} m {}^{m_{010}} m $\\
  528 &  &  & $\overline{6}m2$ & ${}^{3^{+}} 6  / {}^{m_{001}} m {}^{m_{100}} m {}^{m_{010}} m $\\
  529 &  &  &  & ${}^{3^{+}} 6  / {}^{m_{001}} m {}^{2_{1\overline{1}0}} m {}^{2_{120}} m $\\
  530 &  & $1$ & $6/mmm$ & ${}^{6^{+}} 6  / {}^{\overline{1}} m {}^{m_{100}} m {}^{m_{1\overline{1}0}} m $\\
  531 &  &  &  & ${}^{6^{+}} 6  / {}^{m_{001}} m {}^{m_{100}} m {}^{m_{1\overline{1}0}} m $\\
  532 &  &  &  & ${}^{6^{+}} 6  / {}^{\overline{1}} m {}^{2_{100}} m {}^{2_{1\overline{1}0}} m $\\
  533 &  &  &  & ${}^{6^{+}} 6  / {}^{m_{001}} m {}^{2_{100}} m {}^{2_{1\overline{1}0}} m $\\
  534 &  &  &  & ${}^{\overline{3}^{+}} 6  / {}^{m_{001}} m {}^{2_{100}} m {}^{m_{010}} m $\\
  535 &  &  &  & ${}^{\overline{3}^{+}} 6  / {}^{2_{001}} m {}^{m_{100}} m {}^{2_{010}} m $\\
  536 &  &  &  & ${}^{6^{+}} 6  / {}^{\overline{1}} m {}^{2_{100}} m {}^{m_{1\overline{1}0}} m $\\
  537 &  &  &  & ${}^{6^{+}} 6  / {}^{2_{001}} m {}^{m_{100}} m {}^{2_{1\overline{1}0}} m $\\
  538 & $23$ & $23$ & $1$ & ${}^{1} 2 {}^{1} 3 $\\
  539 &  & $222$ & $3$ & ${}^{1} 2 {}^{3^{+}} 3 $\\
  540 &  & $1$ & $23$ & ${}^{2_{001}} 2 {}^{3^{+}_{111}} 3 $\\
  541 & $m\overline{3}$ & $m\overline{3}$ & $1$ & ${}^{1} m {}^{1} \overline{3} $\\
  542 &  & $23$ & $2$ & ${}^{2} m {}^{2} \overline{3} $\\
  543 &  &  & $m$ & ${}^{m} m {}^{m} \overline{3} $\\
  544 &  &  & $\overline{1}$ & ${}^{\overline{1}} m {}^{\overline{1}} \overline{3} $\\
  545 &  & $mmm$ & $3$ & ${}^{1} m {}^{3^{+}} \overline{3} $\\
  546 &  & $222$ & $6$ & ${}^{2} m {}^{6^{+}} \overline{3} $\\
  547 &  &  & $\overline{3}$ & ${}^{\overline{1}} m {}^{\overline{3}^{+}} \overline{3} $\\
  548 &  &  & $\overline{6}$ & ${}^{m} m {}^{\overline{6}^{+}} \overline{3} $\\
  549 &  & $\overline{1}$ & $23$ & ${}^{2_{001}} m {}^{3^{+}_{111}} \overline{3} $\\
  550 &  & $1$ & $m\overline{3}$ & ${}^{m_{001}} m {}^{\overline{3}^{+}_{111}} \overline{3} $\\
  551 & $\overline{4}3m$ & $\overline{4}3m$ & $1$ & ${}^{1} \overline{4} {}^{1} 3 {}^{1} m $\\
  552 &  & $23$ & $2$ & ${}^{2} \overline{4} {}^{1} 3 {}^{2} m $\\
  553 &  &  & $m$ & ${}^{m} \overline{4} {}^{1} 3 {}^{m} m $\\
  554 &  &  & $\overline{1}$ & ${}^{\overline{1}} \overline{4} {}^{1} 3 {}^{\overline{1}} m $\\
  555 &  & $222$ & $32$ & ${}^{2_{1\overline{1}0}} \overline{4} {}^{3^{+}} 3 {}^{2_{1\overline{1}0}} m $\\
  556 &  &  & $3m$ & ${}^{m_{100}} \overline{4} {}^{3^{+}} 3 {}^{m_{100}} m $\\
  557 &  & $1$ & $432$ & ${}^{4^{+}_{001}} \overline{4} {}^{3^{+}_{111}} 3 {}^{2_{1\overline{1}0}} m $\\
  558 &  &  & $\overline{4}3m$ & ${}^{\overline{4}^{-}_{001}} \overline{4} {}^{3^{+}_{111}} 3 {}^{m_{1\overline{1}0}} m $\\
  559 & $432$ & $432$ & $1$ & ${}^{1} 4 {}^{1} 3 {}^{1} 2 $\\
  560 &  & $23$ & $2$ & ${}^{2} 4 {}^{1} 3 {}^{2} 2 $\\
  561 &  &  & $m$ & ${}^{m} 4 {}^{1} 3 {}^{m} 2 $\\
  562 &  &  & $\overline{1}$ & ${}^{\overline{1}} 4 {}^{1} 3 {}^{\overline{1}} 2 $\\
  563 &  & $222$ & $32$ & ${}^{2_{1\overline{1}0}} 4 {}^{3^{+}} 3 {}^{2_{1\overline{1}0}} 2 $\\
  564 &  &  & $3m$ & ${}^{m_{100}} 4 {}^{3^{+}} 3 {}^{m_{100}} 2 $\\
  565 &  & $1$ & $432$ & ${}^{4^{+}_{100}} 4 {}^{3^{+}_{111}} 3 {}^{2_{01\overline{1}}} 2 $\\
  566 &  &  & $\overline{4}3m$ & ${}^{\overline{4}^{-}_{001}} 4 {}^{3^{+}_{111}} 3 {}^{m_{1\overline{1}0}} 2 $\\
  567 & $m\overline{3}m$ & $m\overline{3}m$ & $1$ & ${}^{1} m {}^{1} \overline{3} {}^{1} m $\\
  568 &  & $m\overline{3}$ & $2$ & ${}^{1} m {}^{1} \overline{3} {}^{2} m $\\
  569 &  &  & $m$ & ${}^{1} m {}^{1} \overline{3} {}^{m} m $\\
  570 &  &  & $\overline{1}$ & ${}^{1} m {}^{1} \overline{3} {}^{\overline{1}} m $\\
  571 &  & $\overline{4}3m$ & $2$ & ${}^{2} m {}^{2} \overline{3} {}^{1} m $\\
  572 &  &  & $m$ & ${}^{m} m {}^{m} \overline{3} {}^{1} m $\\
  573 &  &  & $\overline{1}$ & ${}^{\overline{1}} m {}^{\overline{1}} \overline{3} {}^{1} m $\\
  574 &  & $432$ & $2$ & ${}^{2} m {}^{2} \overline{3} {}^{2} m $\\
  575 &  &  & $m$ & ${}^{m} m {}^{m} \overline{3} {}^{m} m $\\
  576 &  &  & $\overline{1}$ & ${}^{\overline{1}} m {}^{\overline{1}} \overline{3} {}^{\overline{1}} m $\\
  577 &  & $23$ & $222$ & ${}^{2_{010}} m {}^{2_{010}} \overline{3} {}^{2_{001}} m $\\
  578 &  &  & $mm2$ & ${}^{m_{100}} m {}^{m_{100}} \overline{3} {}^{m_{010}} m $\\
  579 &  &  &  & ${}^{2} m {}^{2} \overline{3} {}^{m_{010}} m $\\
  580 &  &  &  & ${}^{m_{010}} m {}^{m_{010}} \overline{3} {}^{2} m $\\
  581 &  &  & $2/m$ & ${}^{\overline{1}} m {}^{\overline{1}} \overline{3} {}^{m} m $\\
  582 &  &  &  & ${}^{m} m {}^{m} \overline{3} {}^{\overline{1}} m $\\
  583 &  &  &  & ${}^{2} m {}^{2} \overline{3} {}^{m} m $\\
  584 &  &  &  & ${}^{m} m {}^{m} \overline{3} {}^{2} m $\\
  585 &  &  &  & ${}^{2} m {}^{2} \overline{3} {}^{\overline{1}} m $\\
  586 &  &  &  & ${}^{\overline{1}} m {}^{\overline{1}} \overline{3} {}^{2} m $\\
  587 &  & $mmm$ & $32$ & ${}^{1} m {}^{3^{+}} \overline{3} {}^{2_{1\overline{1}0}} m $\\
  588 &  &  & $3m$ & ${}^{1} m {}^{3^{+}} \overline{3} {}^{m_{100}} m $\\
  589 &  & $222$ & $622$ & ${}^{2_{001}} m {}^{6^{+}} \overline{3} {}^{2_{100}} m $\\
  590 &  &  & $\overline{3}m$ & ${}^{\overline{1}} m {}^{\overline{3}^{+}} \overline{3} {}^{m_{120}} m $\\
  591 &  &  &  & ${}^{\overline{1}} m {}^{\overline{3}^{+}} \overline{3} {}^{2_{120}} m $\\
  592 &  &  & $6mm$ & ${}^{2_{001}} m {}^{6^{+}} \overline{3} {}^{m_{100}} m $\\
  593 &  &  & $\overline{6}m2$ & ${}^{m_{001}} m {}^{6^{+}} \overline{3} {}^{2_{1\overline{1}0}} m $\\
  594 &  &  &  & ${}^{m_{001}} m {}^{6^{+}} \overline{3} {}^{m_{100}} m $\\
  595 &  & $\overline{1}$ & $432$ & ${}^{2_{100}} m {}^{3^{+}_{111}} \overline{3} {}^{2_{01\overline{1}}} m $\\
  596 &  &  & $\overline{4}3m$ & ${}^{2_{100}} m {}^{3^{+}_{111}} \overline{3} {}^{m_{01\overline{1}}} m $\\
  597 &  & $1$ & $m\overline{3}m$ & ${}^{m_{100}} m {}^{\overline{3}^{+}_{111}} \overline{3} {}^{\overline{4}^{+}_{100}} m $\\
  598 &  &  &  & ${}^{m_{100}} m {}^{\overline{3}^{+}_{111}} \overline{3} {}^{2_{01\overline{1}}} m $\\
  \hline\hline
\end{longtable}


%%%%%%%%%%%%%%%%%%%%%%%%%%%%%%%%%%%%%%%%%%%%%%%%%%%%%%%%%%%%%%%%%%%%%%%%%%%%%%%
% References
%%%%%%%%%%%%%%%%%%%%%%%%%%%%%%%%%%%%%%%%%%%%%%%%%%%%%%%%%%%%%%%%%%%%%%%%%%%%%%%
\bibliographystyle{unsrt}
\bibliography{reference}


\end{document}
